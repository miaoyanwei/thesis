\clearpage % Start a new page

\chapter{Formative testing survey responses}
\label{appendix:responses}

\begin{table}[h!]
  \centering
  \scriptsize
  \begin{tabular}{ | p{.15\textwidth} | p{.75\textwidth} | }
    \hline  
    \rowcolor{lightgray} \multicolumn{2}{|c|}{Q1. What do you think this website is about?} \\
    \hline
    P1 & suggesting some better ways to save energy at home and decrease the cost of that \\
    \hline
    P2 & Energy saving \\
    \hline
    P3 & Calculating how much energy is used per/sqr \\
    \hline
    P4 & I think it is about helping individuals to understand strategies to save money while supporting climate change. It seems to be a hybrid between educate visitors and sell "green energy" services/products. \\
    \hline
    P5 & getting energy-related information in my household \\
    \hline
  \end{tabular}
  \caption[]{Question 1}
  \label{tab:question_1}
\end{table}

\begin{center}
    \begin{table}[h!]
    \scriptsize
    \begin{tabular}{ | p{.15\textwidth} | p{.15\textwidth} | p{.15\textwidth} | p{.15\textwidth} | p{.15\textwidth} | }
      \hline  
      \rowcolor{lightgray} \multicolumn{5}{|c|}{Q2. How clear were the instructions on the website for you to follow?} \\
      \hline
      P1 & P2 & P3 & P4 & P5 \\
      \hline
      10/10 & 8/10 & 8/10 & 7/10 & 5/10 \\
      \hline
    \end{tabular}
    \caption[]{Question 2}
    \label{tab:question_2}
    \end{table}
\end{center}

\begin{center}
  \begin{table}[h!]
  \scriptsize
  \begin{tabular}{ | p{.15\textwidth} | p{.15\textwidth} | p{.15\textwidth} | p{.15\textwidth} | p{.15\textwidth} | }
    \hline  
    \rowcolor{lightgray} \multicolumn{5}{|c|}{Q3. Did you find the website visually appealing?} \\
    \hline
    P1 & P2 & P3 & P4 & P5 \\
    \hline
    10/10 & 10/10 & 6/10 & 8/10 & 4/10 \\
    \hline
  \end{tabular}
  \caption[]{Question 3}
  \label{tab:question_3}
  \end{table}
\end{center}

\begin{center}
  \begin{table}[h!]
  \scriptsize
  \begin{tabular}{ | p{.15\textwidth} | p{.15\textwidth} | p{.15\textwidth} | p{.15\textwidth} | p{.15\textwidth} | }
    \hline  
    \rowcolor{lightgray} \multicolumn{5}{|c|}{Q4. Was the website easy to use and understand?} \\
    \hline
    P1 & P2 & P3 & P4 & P5 \\
    \hline
    10/10 & 8/10 & 7/10 & 5/10 & 4/10 \\
    \hline
  \end{tabular}
  \caption[]{Question 4}
  \label{tab:question_4}
  \end{table}
\end{center}

\begin{center}
  \begin{table}[h!]
  \scriptsize
  \begin{tabular}{ | p{.15\textwidth} | p{.15\textwidth} | p{.15\textwidth} | p{.15\textwidth} | p{.15\textwidth} | }
    \hline  
    \rowcolor{lightgray} \multicolumn{5}{|c|}{Q5. How long do you think it took you to complete the questions? } \\
    \hline
    P1 & P2 & P3 & P4 & P5 \\
    \hline
    1-5 min & 1-5 min & 1-5 min & 5-10 min & 1-5 min \\
    \hline
  \end{tabular}
  \caption[]{Question 5}
  \label{tab:question_5}
  \end{table}
\end{center}

\begin{center}
  \scriptsize
  \begin{longtable}[h!]{ | p{.15\textwidth} | p{.75\textwidth} | }
    \hline  
    \rowcolor{lightgray} \multicolumn{2}{|c|}{Q6. What would you change about the website to make it more user-friendly?} \\
    \hline
    P1 & adding a "back" button, in case of returning to the previous page to edit something \\
    \hline
    P2 & 1. I didn't realize if one step moved to the next in the tracker (top left), make you could use a color gradient (i.e. the circles go from light to dark green gradually)  to highlight the progress. 2. Is a little weird that the Children's age is 0-25 (is that the standard in Germany?). 3. It would be great if, in the drop-down menus, there is an "I don´t know" option. And then provide some guidance for the users to find that out (I saw you already have some questions to support the user, I think that's very helpful!). \\
    \hline
    P3 & More explanation, cues \\
    \hline
    P4 & "I felt the need for a back button on the interface. For instance, when I clicked on "more details" on the last page, I couldn't go back to check the other options. I ended up clicking on the logo that lead to the start of the questionnaire.
    I also had to Google the PV meaning. It would be more clear if it was written photovoltaic system. I saw there was a link to explain what PV is, but I think it would be more clear for me if it was written photovoltaic because I know what that means. 
    The steps tracker was not that useful as well. It was not reflecting the number of questions. So I was not sure how many questions would be asked until moved to the next step.
    I am also concerned about the question of the max and min home temperature for me. I never know that as I don't measure it in my home. I would prefer the questionnaire to provide me with a suggestion based on the "ideal" temperature. 
    I don't know if I would decide on an option only by the website usage. Maybe I would like to see more info about the final investment and how much time it would be required to "get that money back" by saving energy consumption from the power provider.
    In the PV system, I would like to be able to see how many I would be able to add to my home to understand how much energy it could generate. At first glance, 3.321 kwh seems to be not much.
    The graphic comparing the current and possible options is not clear. What does it represent? Are the green bars showing how much the PV would generate? Maybe rather than showing many elements (Heating, cooling etc) It would be easier to understand if it don't show that information too granular." \\
    \hline
    P5 & "This is difficult to explain in writting. I would rather speak about this. However, here are few things that can be communicated in a written form. 
    The start pages looks nice! but it can be further improved to make it more appealing and gives better vibes. If this was an interview, I would have showed some examples of what I think would improve it. 
    The questions seemed more like a normal survey. I would rather design it so that it looks more like a friendly inquiry rather than a very serious questionnaire. I would include a more friendly language or even use some slang. Also I would  include few emojies or even illustations where appropriate.
    The 'please wait' page after the questions, gives the impression that the page is not responsive anymore. A more dynamic/moving illustration is expected to know that something is happening and avoid the feeling that the page is lagging. 
    Am I supposed to know information about the battery and PV systems in my house? I was asked for these informations and I am not sure where can I get this information from, if I don't know it. 
    'What is a PV system' and 'What is a battery system' is not active. So I couldn't understand what is that." \\
    \hline
  \caption[]{Question 6}
  \label{tab:question_6}
  \end{longtable}
\end{center}

\begin{center}
  \begin{table}[h!]
  \scriptsize
  \begin{tabular}{ | p{.15\textwidth} | p{.15\textwidth} | p{.15\textwidth} | p{.15\textwidth} | p{.15\textwidth} | }
    \hline  
    \rowcolor{lightgray} \multicolumn{5}{|c|}{Q7. Were the recommendations easy to understand?} \\
    \hline
    P1 & P2 & P3 & P4 & P5 \\
    \hline
    10/10 & 9/10 & 7/10 & 5/10 & 9/10 \\
    \hline
  \end{tabular}
  \caption[]{Question 7}
  \label{tab:question_7}
  \end{table}
\end{center}

\begin{table}[h!]
  \centering
  \scriptsize
  \begin{tabular}{ | p{.15\textwidth} | p{.75\textwidth} | }
    \hline  
    \rowcolor{lightgray} \multicolumn{2}{|c|}{Q8. Was there anything about the recommendations that you found confusing or unclear?} \\
    \hline
    P1 & about PV or SEM systems which I could not see what they are \\
    \hline
    P2 & I love your data visualisation! I would make sure all axis have their respective unit of measure. Just to be extra clear \\
    \hline
    P3 & No \\
    \hline
    P4 & Yes, the bar chart. I think it is also important to understand more clearly the cost of each suggestion and the time to implement such a system. \\
    \hline
    P5 & "In the 'recommendation configuration' page, the word 'current' at the top left is not very clear. I stopped for a second and looked at the information below to know what 'current' referes to here. Also, for a first glance, I was expecting a 'results' page, before the recommendation appears. Here, all is presented in one page.
      For the second page, the axes in the 'energy use' bar chart needs to be named. Also, the annual energy bill is the same for all options. I think it's a typo here..
      Other than that, I think the follow in which the information is presened could be improved." \\
    \hline
  \end{tabular}
  \caption[]{Question 8}
  \label{tab:question_8}
\end{table}
