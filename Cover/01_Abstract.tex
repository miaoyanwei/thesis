\pagenumbering{roman} \setcounter{page}{3}
\begin{center}
{\Large{\bf{ABSTRACT}}}
\end{center}

\noindent

The transition to clean energy and energy-efficient technologies is crucial for reducing carbon emissions and mitigating climate change. 
However, households lack sufficient knowledge and guidance on these technologies, including the potential benefits that can be obtained through their adoption.
This study aims to fill the information gap and support decision-making on the adoption of clean energy and energy technologies for homeowners by introducing a novel IT artefact, a personalised home energy system recommender.
Meanwhile, this study also seeks to develop effective explanations within a recommender system that fosters trust and facilitates knowledge acquisition, thereby assisting households in making sustainable decisions.
This study employs Design Case Studies within the framework of Grounded Design. 
Pre-study activities involved online searches and literature reviews. 
The subsequent concept development phase focused on a learning theory-aligned approach. 
Following the online deployment of the service, a qualitative user study assessed its effectiveness, emphasising two critical aspects: 
Knowledge Enhancement: Evaluating users' augmentation of understanding regarding energy technologies and their benefits.
Trust in Recommendations: Building users' confidence in the system's suggestions.
Results, expounded upon in the discussion section, underscore the significance of financial considerations, motivations, trust, and the need for comprehensive information. 


\dots
%A website software will be developed to provide households with personalised and professional home energy system recommendations and techno-economic assessments. 
%The interfaces will be designed based on the findings from a pre-study conducted through literature review. 
%Real household users will use the intervention and provide feedback through both qualitative and quantitative data collection methods. 
%This research will contribute to the HCI community by providing an understanding of how technology can be used to promote the adoption of clean energy and energy technologies in households, 
%and by identifying the effectiveness of personalised recommendations in supporting behaviour change. 
\clearpage