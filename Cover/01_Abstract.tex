\pagenumbering{roman} \setcounter{page}{3}
\begin{center}
{\Large{\bf{ABSTRACT}}}
\end{center}

\noindent

The transition to clean energy and energy-efficient technologies is crucial for reducing carbon emissions and mitigating climate change. 
However, households lack sufficient knowledge and guidance on these technologies, including the potential benefits that can be obtained through their adoption.
This study aims to fill the information gap and support decision-making on the adoption of clean energy and energy technologies for homeowners by introducing a novel IT artefact, a personalised home energy system recommender.
Meanwhile, this study also seeks to develop effective explanations within a recommender system that fosters trust and facilitates knowledge acquisition, thereby assisting households in making sustainable decisions.

This study employs Design Case Studies within the framework of Grounded Design. 
Pre-study activities involved online searches and literature reviews,
aims to investigate existing practices and tools conducive to homeowners acquiring knowledge about renewable energy and energy-efficient technologies, including their benefits.
The subsequent concept development phase focused on a constructivism-aligned approach. 
There has been two evaluations performed in this study, formative evaluation before actual programming and summative evaluation after the online deployment of the service.
Qualitative user studies were conducted during the summative evaluation to assess the artefact's effectiveness, emphasising two critical aspects: 
Knowledge Enhancement: Evaluating users' augmentation of understanding regarding energy technologies and their benefits.
Trust in Recommendations: Building users' confidence in the system's suggestions.

The results highlight that the provision of information significantly influences homeowners' decision-making, with slight variations in user responses. 
Different user backgrounds require distinct levels of explanations; for instance, energy professionals expect more detailed insights into the technical intricacies of the model and algorithms.
The design of the artefact prioritised explainability to build trust among users. 
Homeowners, in general, expressed confidence in the recommendations provided. 
The service offered three main levels of explanations, with financial and usage pattern explanations being well-received by users. 
On the other hand, educational explanations focusing on climate change appear less appealing to many.
Furthermore, the results offer insights into financial considerations, motivations, trust, personalisation concerns, and the need for comprehensive information. 
In essence, this study unveils multifaceted insights through the lens of the developed IT artefact, providing an understanding of various dimensions related to homeowners' decision-making in adopting energy technologies.


%A website software will be developed to provide households with personalised and professional home energy system recommendations and techno-economic assessments. 
%The interfaces will be designed based on the findings from a pre-study conducted through literature review. 
%Real household users will use the intervention and provide feedback through both qualitative and quantitative data collection methods. 
%This research will contribute to the HCI community by providing an understanding of how technology can be used to promote the adoption of clean energy and energy technologies in households, 
%and by identifying the effectiveness of personalised recommendations in supporting behaviour change. 
\clearpage