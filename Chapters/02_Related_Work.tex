\chapter{Related work} 


%Within the academic community, there exists a concerted effort to address issues related to energy efficiency and the adoption of renewable energy sources. 
%There are numerous research and ongoing projects aimed at establishing a robust social energy infrastructure capable of adapting to the utilisation of renewable energy sources. 
%Studies are also investigating the feasibility and practicalities of establishing zero-emission households and buildings.
%Technological innovations designed to support these goals have also been a focal point in the academic discourse surrounding energy efficiency and the transition to renewable energy.

%In the meantime, Palmer et al. \cite{informationgap} drew attention to the fact that engineering studies have identified various investments in new energy-efficient equipment or building retrofits that would generate savings surpassing their costs in terms of lower future energy expenses. However, homeowners and businesses lack sufficient knowledge and guidance on how to effectively utilize these opportunities to their advantage.

%Empirical results suggest that households' propensity to invest in clean energy technologies depends mainly on home ownership, income, social context and household energy conservation practices,
%in addition, environmental attitudes and beliefs, as manifest in energy conservation practices or membership in an environmental non-governmental organisation, also play a relevant role in technology adoption \cite{determinants}.

This section initiates by examining the current landscape of available methodologies and potential resources enabling homeowners to access information regarding energy technologies, their associated benefits, and other relevant data. 
This examination serves as a driving force behind the creation of an innovative design concept, seeking to address potential limitations inherent in current approaches.
In addition, the significance of theoretical frameworks cannot be understated in shaping design concepts and understanding user behaviour within the HCI domain. 
Constructivism stands as a guiding principle in informing the development of this study.
Expanding upon this exploration, the section delves into the domain of recommendation systems. 
This study aims to underscore the impact of personalised, user-driven recommendation systems, shedding light on their substantial contribution to user learning, trust, and decision-making within the domain of energy technologies and sustainable household practices.


\section{Current practices and potentials}

Currently, homeowners have limited avenues to access information about home energy systems. 
Individuals seeking such information typically have two options. 
One is visiting specific technology providers,
an alternative approach is through professional home energy assessments. 
Furthermore, research-based models offer evidence to aid homeowners in making informed decisions regarding home energy systems.


\subsection{Technology providers}

Exploring the official websites of technology providers or visiting nearby stores specialising in energy technologies can indeed provide valuable information about specific technologies. 
However, this necessitates a prior knowledge of the particular energy technology.
Moreover, the information obtained through this approach may be restricted to the specific technology being explored, 
thus failing to offer a holistic perspective on the overall energy system, as energy technologies often function collaboratively.


\subsection{Energy audits}

Professional home energy assessments, commonly known as home energy audits.
These assessments are conducted by experts who visit the house and perform a comprehensive inspection. 
Following the assessment, these professionals provide recommendations regarding house renovations and advice on suitable energy technologies to optimise energy efficiency. 

Germany has a wide network of advisory centers and municipal institutions, totaling around 740, that provide energy advice to private households \cite{bmwk2023}. 
These centers offer various services aimed at helping households optimise their energy usage and reduce costs. 
One prominent example is the Verbraucherzentrale Energieberatung \cite{VB2023}, 
which offers independent energy consultants, including individual energy advice and funding tips. 
The advice provided by these centers covers various important topics, including saving electricity in households, tips for energy conservation as tenants, guidance on thermal insulation and summer heat protection, advice on proper heating and ventilation practices, insights into renewable energy options, and information on modern heating technologies.
Notably, these services are public funded, ensuring affordability for the general public, with consultation fees capped at a maximum of 30 euros. 
Moreover, low-income households can access these services free of charge.
Furthermore, some advisory centers also provide online consultations for an initial assessment and to address specific energy-related inquiries. 
However, the primary focus of these services remains on providing in-person, on-spot consultations. 

\subsection{PVGIS online tool}
PVGIS \cite{pvgis} is a web-based application by the European Commission's Joint Research Centre, 
that enables users to access comprehensive data regarding solar radiation and the energy production of \gls{pv} systems. 
This service encompasses a wide range of geographical regions, including Europe, Africa, substantial portions of Asia, and America.
Which can be of a great help to house owners when deciding an investment in a \gls{pv} system. 

As shown in the Figure \ref{fig:pvgis}, the interactive tool allows users to navigate through the map and obtain information regarding performance of grid-connected \gls{pv} based on the selected location. 
The visualisation of monthly energy output provides a clear and descriptive representation of the energy generated by a \gls{pv} system throughout a year. 
Additionally, the outcome offers highly precise and specialised data, including detailed parameters such as yearly in-plane irradiation and year-to-year variability as well. 
While this information is highly valuable for researchers, it may pose comprehension challenges for homeowners lacking expertise in the field, thereby hindering their learning process. 
Furthermore, the data provided is only \gls{pv} related, lacking the connection to the specific circumstances of individual households. 
\begin{figure}[h!]
  \centering
  \includegraphics[width=\textwidth]{Images/pvgis.png}
  \caption{Screen of PVGIS online tool}
  \label{fig:pvgis}
\end{figure}


\subsection{FLEX models}

The FLEX models \cite{newtrends}, developed under the newTRENDs project\footnote{https://newtrends2020.eu/} by the Fraunhofer Institute for Systems and Innovation Research, 
aim to improve the building modeling suite and to analyse the societal trends of prosumaging and energy communities, 
are capable of calculating the energy demand of buildings at an hourly resolution,
while considering the impact of household behaviour, \gls{pv} generation, and energy storage (thermal and battery) on energy consumption. 
These models were developed to offer evidence-based information to decision-makers in industry, government, and civil society. 
%By providing a comprehensive assessment of the impacts of emerging technologies and innovation strategies, these models enable stakeholders to make informed decisions concerning policies related to technology and innovation. 

The models take various factors into account, including weather condition, household behaviours and energy technologies, as illustrated in Figure \ref{fig:flex-operation}.
Consequently, it offers a comprehensive evaluation of the energy consumption of a building. 
Moreover, the tool can be used to predict energy bills, enabling comparisons of energy expenses associated with different technology adoptions. 

%The Figure \ref{fig:flex} shows how FLEX interacts with other bottom-up models involved in the newTRENDs project,
%where FLEX-Operation and FLEX-Behaviour models are closely related to the demand and supply of energy for households. 
%\begin{figure}[h]
%  \centering
%  \includegraphics[width=\textwidth]{Images/flex.png}
%  \caption{FLEX modeling suite}
%  \label{fig:flex}
%\end{figure}

%Renewable energy (\gls{re}) and energy efficiency (\gls{ee}) are two central strategies pursued by the \gls{eu} and its Member States concerning the energy system. 
%In 2019, 80.9\% of our total energy supply still depended on burning fossil fuels, namely 26.8\% coal, 30.9\% oil and 23.2\% natural gas \cite{iea}. 
%Investments into low-carbon power generation accounted for 15\% recently are expected to rise to more than 30\% by 2030, corresponding to a quadrupling in absolute volumes \cite{shift}. Solar, wind, and the investments for enabling the integration of these technologies to the grid dominate the investments into low-carbon power generation \cite{shift}. 
%Electrification is playing a major role in the energy transition process. 
%Meanwhile, different electrification strategies rely heavily on energy efficiency \cite{electrification}.
%Measures to increase energy efficiency, including investments in energy savings and the consolidation of consultancy and information services, are promoted by The National Action Plan on Energy Efficiency (\gls{nape}) \cite{bafa}.  

%Transitioning towards a sustainable energy system necessitates significant effort on both the demand and supply sides. 
%However, previous research has shown that in many areas energy efficiency gains were counteracted by societal trends that increased corresponding activities, leading to much smaller decreases (or even increases) of energy demand than technologically feasible \cite{2050}. 
%The aim of newTRENDs is to increase the qualitative and quantitative understanding of impacts of new societal trends on energy consumption and to improve the modelling of energy demand, energy efficiency and policy instruments \cite{fraunhofer}. 


%\subsubsection{The FLEX-Operation model}

%The FLEX-Operation model \cite{newtrends} enables the detailed simulation of energy system operation for individual households at an hourly resolution. 
%This model provides a comprehensive assessment of the energy consumption of a representative building, incorporating technology operation (such as battery, \gls{pv}, and \gls{hp} systems) and load profiles at an hourly resolution. 
%In addition to its capability of modeling energy system operation, FLEX-Operation can also aid in investment decision-making by evaluating the energy-saving benefits associated with technology adoption.
%
%As shown in Figure \ref{fig:flex-operation}, FLEX-Operation considers following services:

%\begin{enumerate}
%  \item electric appliances, e.g., television, refrigerator, lighting, etc.;
%  \item space heating;
%  \item domestic hot water;
%  \item space cooling;
%  \item vehicle. 
%\end{enumerate}

\begin{figure}[h!]
  \centering
  \includegraphics[width=\textwidth]{Images/flex-operation.png}
  \caption{Model structure for individual households}
  \label{fig:flex-operation}
\end{figure}

%Researchers believe new societal trends have the potential to shift energy demands between sectors and might reinforce or diminish one another when they occur at the same time \cite{2050}. 
%Researchers and organisations are paying increasing attention to how new societal trends are affecting energy demand.
%It is therefore important to access current and (foreseeable) future societal trends concerning the impact that they might have on future energy demand \cite{2050}. 

%Four arising societal trend clusters that are likely to shape future energy demand in European countries (and worldwide) were established by Brugger et al. \cite{2050}:  
%\emph{
%  (1) the digitalization of the economy and of private life; 
%  (2) new social and economic models, including the sharing economy and prosumaging (combination of producing, consuming and managing of energy); 
%  (3) industrial transformation, including decarbonization of industrial processes and the circular economy (including a stronger focus on material efficiency); 
%  (4) quality of life, including health effects, urbanization and regionalization. 
%}
%
%\begin{itemize}
%  \item \textbf{Digitalization of life} %\\ the digitalization of the economy and of private life;
%  \item \textbf{New social and economic models} %\\ including the sharing economy and prosumaging (combination of producing, consuming and managing of energy);
%  \item \textbf{Industrial transformation} %\\ including decarbonization of industrial processes and the circular economy (including a stronger focus on material efficiency);
%  \item \textbf{Quality of life} %\\ including health effects, urbanization and regionalization. 
%\end{itemize}
%
%The newTRENDs project develops the analytical basis for a “2050 Energy Efficiency Vision” by considering new societal trends in energy demand modeling \cite{newtrends}. 
%Considering the impact of these new societal trends on energy demand from a closer sectoral perspective,
%Yu et al. \cite{newtrends} identified four energy sectors: 
%
%\begin{itemize}
%  \item industry, 
%  \item transport,  
%  \item tertiary, 
%  \item residential.  
%\end{itemize}

%This proposed thesis will focus on the residential sector while taking scenarios of “consumers” becoming “prosumers” (with \gls{pv}) and “prosumagers” (adding energy storage and \gls{sems}) \cite{consumer} into account.  



%\subsubsection{The FLEX-Behaviour model}

%The FLEX-Behaviour model \cite{newtrends} facilitates the modeling of household behavior, including activity profiles and corresponding load profiles. 
%By generating an hourly activity and energy demand profile for a pre-defined household, this model provides a comprehensive assessment of the energy consumption patterns of an individual household.
%
%The estimates generated by the FLEX-Behaviour model refer to: 
%\begin{itemize}
%  \item appliance electricity demand,
%  \item domestic hot water demand,
%  \item driving profile, and
%  \item building occupation.
%\end{itemize}

%\subsubsection{INVERT/EE-Lab and FORECAST-Appliance}
%
%INVERT/EE-Lab and FORECAST-Appliance are the two models that can cover the energy consumption of residential buildings. The two models complement each other and cover the total energy consumption of households. 
%However, both INVERT/EE-Lab and FORECAST-Appliance calculate the energy consumption at the annual resolution and cannot model the prosumaging behavior and energy community, which requires an hourly resolution to consider the impact of household behavior, \gls{pv} generation, and energy storage (thermal and battery) on energy consumption. 
%In this regard, the FLEX-Operation and FLEX-Community models were developed to improve the building modeling suite and support relevant policy analysis \cite{newtrends}. 

%\subsubsection{FLEX-Community}
%
%FLEX-Community models the operation of an energy community, i.e., household interaction, aggregator optimisation. 
%It can be applied to support the aggregators designing and evaluating business models, as well as making investment decisions, for example, the self-owned battery, \gls{pv} panels, etc. \cite{newtrends}.


It is important to note that the FLEX models are designed to estimate for single building structures, meaning buildings that do not share walls with other buildings. 
Furthermore, the FLEX models are implemented in Python. 
To execute these models, users would need to have Python installed on their systems. 
Additionally, since the FLEX models involve complex optimisation problems, a solver is required. 
Users would need to download and set up the appropriate solver to run the FLEX models effectively.
Moreover, the outputs generated by the FLEX models are in the form of SQL files, 
which might not be immediately interpretable to non-technical users.


\section{Relevant theories}

The understanding of energy technologies and their implications in domestic settings can be regarded as a form of knowledge claims \cite{Bolisani2018}.
Given this, the process of obtaining this knowledge can be likened to a learning journey.
To enhance the dissemination of this knowledge to homeowners, learning theories are integrated into the design concept.


\subsection{Constructivism}

Constructivism, a psychological learning theory, delineates how individuals acquire knowledge and learn. 
It regards the learner as an active participant in knowledge assimilation.
This theory emphasises that individuals construct knowledge and meaning from their experiences. 
It advocates for an approach to teaching and learning where cognition (learning) is the result of "mental construction",
in essence, students learn by connecting new information to their existing knowledge \cite{Bada2015}. 

The integration of constructivist concepts aims to bridge the information gap by tailoring personalised recommendations to the unique contexts of individual households.
These theoretical frameworks propose that users are more inclined to grasp information about technologies and their advantages when they can relate the knowledge to their specific home energy systems.
By aligning information with their unique situations, users are more likely to engage and retain knowledge due to the contextual relevance of the recommendations.


\section{Transition to recommendation systems}

Recommender systems (\gls{rs}) are software tools and techniques designed to provide users with suggestions for items that might be of interest, 
assisting in various decision-making processes \cite{Ricci2011}. 
\gls{rs}s primarily target individuals who may lack the personal experience or expertise to evaluate an extensive range of alternatives \cite{Ricci2011}. 
In the context of homeowners seeking insights into energy technologies, 
\gls{rs} becomes particularly valuable, aiding those who are not well-versed in this domain to navigate the vast array of available options.

Building upon Constructivist theories of learning \cite{Bada2015}, 
this study emphasises a user-centric approach, aiming to bridge the information gap for homeowners.
Incorporating the learning theory, the focus is on linking new knowledge to existing experiences. 
Through \gls{rs} integration, users can merge the new knowledge of energy technologies with their present home energy systems, 
enabling a more comprehensive understanding and fostering a contextualised learning process.
Integrating \gls{rs} into the design concept aligns with the principles of Constructivism,
as it empowers users by enabling them to construct their understanding by associating new energy technology knowledge with their current home systems.

\gls{rs}s have a variety of properties that may affect user experience, 
such as accuracy, robustness, scalability, and so forth \cite{Shani2011}. 
However, in this study, our primary focus lies on the aspect of trustworthiness, 
recognised as an important consideration in \gls{rs} design \cite{Donovan2005}.

Drawing from Kirsten Swearingen and Rashmi Sinha's research \cite{rs},
specific recommendations are suggested to cultivate trust within a recommendation system.
Their proposed measures aim to promote trust by:
\emph{
  ensuring transparent system logic; 
  suggesting novel items; 
  offering comprehensive information about recommended items; 
  and enabling users to refine their recommendations by specifying preferred or excluded genres.
}

In summary, the integration of \gls{rs} within the design concept stands as an attempt to bridge the information gap 
by combining theoretical learning frameworks with a personalised recommendation system. 
This aims to enhance user learning experiences and foster more informed decisions regarding sustainable energy technologies for households.


\section{Sustainable energy systems}

The pursuit of sustainable energy systems is integral to addressing the urgent challenges posed by climate change and fostering environmentally responsible societies. 
At its core, the concept hinges on three key activities outlined by Matias et al. \cite{Matias2020}:
\begin{enumerate}
  \item \textbf{Increased use of renewable energy sources:} A fundamental step involves a substantial shift towards incorporating renewable energy sources into the overall energy mix. This includes harnessing energy from environmentally friendly sources such as solar, wind, hydro, and geothermal.
  \item \textbf{Enhanced energy efficiency:} Energy efficiency stands as a cornerstone policy, emphasising the optimisation of energy consumption across various sectors. This involves employing technologies and practices that minimise energy waste while maximising output, contributing to overall sustainability.
  \item \textbf{Emission reduction:} A critical aspect involves mitigating the emissions of greenhouse gases and air pollutants. Łukasiewicz et al. \cite{Lukasiewicz2022} propose a conceptual model supporting these efforts, offering a framework for in-depth analyses in the pursuit of sustainable energy solutions.
\end{enumerate}

Long-term actions for Sustainable Development (\gls{sd}) are vital for addressing environmental concerns. 
Renewable energy (\gls{re}) resources emerge as a particularly promising and efficient solution, playing a pivotal role in achieving environmental sustainability \cite{Shoeib2021}.

In summary, the realisation of sustainable energy systems necessitates a concerted effort to transition towards renewable energy, optimise energy usage, and curtail harmful emissions, all contributing to the overarching goal of building a sustainable and resilient future.


\section{Conclusion}

%financial aspects play a significant role in guiding homeowners' decisions when considering upgrades to their home energy systems. 
Our exploration of the current practices and potentials within the domain of energy technologies and homeowner accessibility to information has revealed a noticeable information gap. 
Homeowners face limitations in obtaining comprehensive insights into energy technologies and their potential benefits.
The current approach, energy audits, can be quite laborious, 
involving booking appointments with experts, conducting house inspections, and investing considerable time in the assessment process. 
While government financial support might make the audits affordable, 
the effort required can discourage homeowners from seeking information and exploring energy-efficient options for their homes.

On the other hand, research-based models, particularly the FLEX models, 
are powerful tools that offer valuable insights through their detailed data and precise estimations of energy consumption and associated energy bills. 
These models consider various factors, including the house's characteristics, location, and energy technologies configuration. 
However, running these models demands professional knowledge, making them less accessible to general homeowners.
This creates an opportunity for the development of an innovative IT artefact that could revolutionise the way homeowners access information about energy technologies. 

The foundational principles of constructivism demonstrate the importance of contextual relevance in understanding and retaining knowledge.
These theories underpin the approach of personalising recommendations to individual home situations.
This artefact aims to empower homeowners with easy access to information, enhancing their understanding and fostering informed decisions regarding sustainable energy technologies for their homes.