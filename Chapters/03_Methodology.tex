\chapter{Methodology} 

The study uses Design Case Studies \cite{dcs} as the research framework. 

In the pre-study phase, 
the primary focus was to investigate existing practices and tools that can aid homeowners in gaining knowledge about renewable energy and energy-efficient technologies, as well as their benefits. 
To achieve this, we conducted extensive searches online and reviewed relevant literature to gather information. 
However, as there were limited successful initiatives available in the market, we identified two related options: energy audits and research models. 
We studied both options to identify effective methods that can support homeowners in understanding renewable energy and energy-efficient technologies, along with their associated benefits.

Following the pre-study, 
we recognised a suitable approach aligned with the learning theory, 
and developed an innovative design concept: a personalised home energy system recommender.
To ensure its effectiveness, 
we started by investigating homeowners' motivations for investing in energy technologies through literature. 
Next, our focus was on providing recommendations that are aligned with user needs, 
and we placed great emphasis on enhancing the explainability of the system.

Furthermore, during the IT artefact design process, 
multiple factors were taken into account, including usability, user experience, and the chosen medium. 
Additionally, 5 testings were conducted on high-fidelity wireframes, 
which helped identify some issues that were then addressed through design iterations. 

After the service was programmed and made available online, 
we conducted an investigation to assess the appropriation of the artefact,
with a specific focus on two aspects:
\begin{enumerate}
    \item Do users enhance their knowledge about energy technologies and the advantages of their adoption through the service?
    \item Do users have confidence in the recommendations provided by the system?
\end{enumerate}
6 qualitative evaluations were performed with 7 actual house owners. 
These evaluations were conducted through semi-structured interviews, 
each lasting approximately one to three hours. 

After conducting the evaluations, a thematic analysis was performed to gain deeper insights into users' experiences and perceptions. 
This analysis provided valuable feedback and insights that can be used for the next design iteration, allowing for further improvements and enhancements to the artefact. 

%The methodology adopted in this study is based on the Design Case Studies framework. 
%The pre-study phase will begin by conducting a comprehensive review of the literature to identify best practices for providing households with personalised and professional home energy system recommendations, as well as techno-economic assessments.
%Based on the findings from the pre-study, I will then design the interfaces of the intervention. 
%The interfaces will be developed to provide an intuitive and user-friendly experience that can easily be understood by households. 
%Following the development phase, real users will be invited to use the intervention, and feedback will be collected both qualitatively and quantitatively. 
%The qualitative data will be collected through interviews with participants, while the quantitative data will be collected through surveys. 
%Finally, the collected data will be analysed to evaluate households thoughts about the recommendations and energy technologies. 
%The entire process will be documented and reported in the form of a thesis. 
