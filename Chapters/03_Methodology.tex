\chapter{Methodology}

The study adopts Design Case Studies \cite{dcs} as the research framework, with a specific emphasis on Grounded Design \cite{Stevens2018}.

In the pre-study phase, our primary objective was to investigate existing practices and tools conducive to homeowners acquiring knowledge about renewable energy and energy-efficient technologies, including their benefits. 
To achieve this, we conducted comprehensive online searches and reviewed relevant literature. Despite limited successful initiatives in the market, we identified two pertinent options: energy audits and research models. 
Both were scrutinized to discern effective methods for supporting homeowners in comprehending renewable energy and energy-efficient technologies, along with their associated benefits.

Subsequent to the pre-study, a suitable approach aligned with learning theory was identified, leading to the development of an innovative design concept: a personalised home energy system recommender. 
To ensure its efficacy, we investigated homeowners' motivations for investing in energy technologies through existing literature. 
Our focus then shifted to providing recommendations that align with user needs, with a particular emphasis on enhancing the explainability of the system.

Throughout the IT artefact design process, multiple considerations were factored in, including usability, user experience, and the chosen medium. 
Additionally, five formative evaluations were conducted on high-fidelity wireframes, aiding in the identification of issues that were subsequently addressed through design iterations.

Once the service was programmed and made available online, a summative investigation was conducted to assess the appropriation of the artefact, with specific emphasis on two aspects:

\begin{enumerate}
    \item \textbf{Knowledge Enhancement:} Do users augment their knowledge about energy technologies and the advantages of their adoption through the service?
    \item \textbf{Trust in Recommendations:} Do users have trust in the recommendations provided by the system?
\end{enumerate}

Six qualitative evaluations were performed with seven actual homeowners. 
These evaluations took the form of semi-structured interviews, each lasting approximately one to three hours.

Following the evaluations, a thematic analysis was conducted to delve deeper into users' experiences and perceptions. 
This analysis yielded valuable feedback and insights, serving as a foundation for the next design iteration and allowing for continuous improvements and enhancements to the artefact.

%The methodology adopted in this study is based on the Design Case Studies framework. 
%The pre-study phase will begin by conducting a comprehensive review of the literature to identify best practices for providing households with personalised and professional home energy system recommendations, as well as techno-economic assessments.
%Based on the findings from the pre-study, I will then design the interfaces of the intervention. 
%The interfaces will be developed to provide an intuitive and user-friendly experience that can easily be understood by households. 
%Following the development phase, real users will be invited to use the intervention, and feedback will be collected both qualitatively and quantitatively. 
%The qualitative data will be collected through interviews with participants, while the quantitative data will be collected through surveys. 
%Finally, the collected data will be analysed to evaluate households thoughts about the recommendations and energy technologies. 
%The entire process will be documented and reported in the form of a thesis. 