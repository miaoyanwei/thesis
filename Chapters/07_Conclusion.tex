\chapter{Conclusion}

Our exploration to bridge the gap in knowledge about energy technologies and the associated benefits for homeowners has yielded valuable insights. 

The journey began with an initial phase of pre-study, where our goal was to uncover existing tools and practices that could empower homeowners to embrace sustainable energy solutions. 
This initial investigation paved the way for the development of an innovative home energy system recommender, anchored in the FLEX models. 
These models assist in identifying technology configurations tailored to specific home situations, leading to potential energy cost savings that directly benefit homeowners financially. 
Through this process, homeowners became acquainted with technologies and their associated benefits. 

Throughout the design process, we placed significant emphasis on explainability, ensuring the recommendations were trusted.  
By engaging with real participants through user testing, we have gained a deeper understanding of their expectations, concerns, and attitudes. 
The insights gathered from the feedback lead us to a conclusive observation.


\subsubsection*{Learning and information gaps}

The provision of information significantly influences homeowners' decision-making, despite slight variations in user responses. 
These differences underscore varying perspectives and positive attitudes towards investing in such technologies. 
Users with limited knowledge about energy technologies before using the service expressed gaining a better understanding of various technologies, their functionalities, and the potential financial benefits.
In addition, participants who were already knowledgeable in this field learned about the impact of different configurations and sizes of technologies tailored to their unique situations. 
While participants with related professional knowledge did not notably enhance their understanding through the service regarding the technologies, those unfamiliar with specific technologies appreciated the general overview but desired more detailed information. 
Thus, the need for additional depth in explanations became apparent.

Furthermore, drawing from participants' feedback, it is evident that the service effectively emphasised the financial perspective, a key element that piqued the interest of homeowners. 
Further insights into homeowners' viewpoints on financial aspects will be explained more below.


\subsubsection*{Trust and confidence in the system}

Moreover, trust within a recommender system, plays a critical role in nurturing user confidence and promoting well-informed decisions.
As discussed in the previous chapter, user feedback has unveiled various factors influencing trust and distrust in both recommendations and the system. 
Our recommender system incorporates three levels of explanations, each soliciting responses from users.
The first level of explanation, which centers on annual energy bills and investment costs, as well as provides a comparison with their estimated current bills, garners appreciation from all users. 
This aspect contributes to building trust in the system, aligning with their financial concerns and enabling a comparison with real-world situations. 
The second level of explanation, visualising energy consumption patterns, is also well-received by users. 
This visualisation resonates with their understanding of consumption habits and further bolsters their trust. 
Additionally, the ability to freely modify configurations and compare results plays a vital role in enhancing trust. 
Witnessing the differences between configurations fosters the belief that the system tailors recommendations to individual situations, rather than offering one-size-fits-all suggestions. 

However, the third level of explanation - the educational information, appears to be less appealing to many users, potentially due to its less user-friendly interaction design involving extensive text. 
Furthermore, as many users are already familiar with the concept of climate change, this section often serves as a self-checklist rather than a trust-building element. 
Consequently, this level of explanation may not significantly contribute to user trust in the recommendations or the system. 
Nevertheless, it serves as valuable supplementary information, allowing users to reflect on their current behaviours and potentially consider more sustainable choices.
Participants expressed trust in the recommendations, citing factors such as accurate energy bill estimations, alignment with professional expertise, and the credibility of the service's source. 
The well-designed interface also contributed to a sense of professionalism and reliability, fostering confidence in the system.


In addition to the aforementioned perspectives that the study aimed to emphasise, the research also delves into insights regarding the following aspects.


\subsubsection*{Participant awareness and background}

The participants demonstrated a collective and robust awareness of climate change. 
Among them, those employed in energy-related fields exhibited significant knowledge, while others derived their insights from varied sources. 
Notably, most participants were either current adopters or considering the installation of energy-efficient technologies in their homes, reflecting a pre-existing interest in energy topics.


\subsubsection*{Financial considerations and motivations}

Furthermore, in alignment with findings from previous studies and surveys, our research highlights the enduring significance of financial considerations as a primary motivating factor. 
Nonetheless, a notable shift is observed in our study, where participants express an increasing desire to contribute to environmental protection.
It is noteworthy that this "green" inclination, while supportive, remains secondary to the financial aspect due to the substantial investment required. 
However, this change in attitude is distinct from the situation observed in a survey conducted in 2013 \cite{informationgap}. 
This evolving perspective suggests a growing awareness and willingness among individuals to embrace more sustainable energy decisions.

Financial aspects remain as pivotal in participants' decision-making processes, with potential cost savings being a primary motivator. 
The transparency of investment costs and estimated energy bills played a crucial role. 
Professionals in the energy sector sought more pprofessional information, revealing their inclination towards in-depth insights for informed decision-making.


\subsubsection*{Personalisation and user concerns}

User feedback highlighted concerns related to the level of personalisation in recommendations. 
Unique circumstances, consumption behaviours, and regional variations in investment costs underscored the need for a balanced approach between simplicity and detail in user interactions.


\subsubsection*{Holistic recommendations and user needs}

Beyond energy technologies, participants expressed interest in recommendations for materials and renovations. 
This suggests a need for a more comprehensive approach to accommodate diverse user preferences and requirements, considering various aspects of energy-efficient enhancements.


\subsubsection*{Government support and policy implications}

Participants emphasised the significant financial commitment involved in adopting sustainable energy systems and called for substantial government support. 
This insight underscores the role of governments in incentivizing and promoting energy-efficient technologies through subsidies or other forms of financial assistance.


\subsubsection*{Multifunctionality and service impact}

Participants recognised the multifunctional potential of the service, viewing it as a possible alternative to the "Energieausweis" in Germany. 
This multifaceted utility adds versatility to the service, potentially expanding its reach and relevance in the energy domain. 


In accordance with Fogg's Persuasive Design framework \cite{Fogg2009}, 
where he outlined how technology can effectively influence and change behaviours by considering three key factors: 
\emph{motivation, prompt, and ability}. 
Our study reveals that a considerable number of households exhibit strong ``motivations" to embrace energy-efficient technologies. 
The introduced home energy system recommender functions effectively as a ``prompt," providing valuable insights and recommendations that facilitate their decision-making processes. 
Although, there is room for improvement in this artefact, as discussed in the previous chapter,
for instance, users with varying levels of knowledge about energy technologies express their specific informational needs.
We envision a promising future for tools like ours.
These tools not only provide valuable information to households, fostering financial benefits and sustainable energy investments, while also potentially aiding in mitigating climate change. 
However, a prevalent ``ability" constraint emerges among many households,
primarily associated with financial challenges that impede one-time investments in these technologies. 
This emphasises the necessity for policy interventions aimed at improving accessibility and affordability. 
We believe that through the provision of accessible information and collaborative efforts from governing bodies, this trend of embracing sustainable energy choices will continue to grow, encouraging more individuals to take meaningful steps toward a greener and more energy-efficient future.  


\section{Next steps}

Based on the preceding analysis, several key directions emerge for future development:

There is a need to provide access to more comprehensive and detailed information regarding the recommended technologies. 
This could involve incorporating additional explanations, or links to external resources to facilitate a deeper understanding of the technologies.

Considering the positive response to the comparison of energy consumption patterns, adding further information like \gls{co2} emission comparison graph could enhance the users' decision-making process, providing them with another perspective on the environmental impacts of their choices.

Personalisation remains a cornerstone for building trust and user engagement. 
Thus, allowing users to input more personalised data related to their specific context and energy consumption patterns could further enhance the accuracy and relevance of recommendations as well as their confidence in the system.

Additionally, incorporating more detailed insights into the calculation of estimated bills, which may encompass factors such as regional subsidies, would provide users with a more comprehensive understanding of the financial implications associated with their choices.

To broaden the service's utility, the potential integration with the existing ``Energieausweis" requirement is worth exploring. 
Providing an energy certificate as part of the service could serve a purpose of facilitating user understanding and fulfilling a regulatory need.

Expanding the applicability of the service to a broader range of house types should also be considered. 
This can make the service more relevant and accessible to a wider audience. 

The third level of explanation could be more interactive and engaging for users. 
Incorporating visual aids, interactive elements, and concise text to enhance the overall user experience and understanding.

Additionally, to ensure continuous improvement, integrating the Kano survey into the service would be important to allow for ongoing feedback collection. 

Moreover, from a technical perspective, enhancing the flexibility of the service is crucial. 
Current hardcoded elements should be replaced with more dynamic components to ensure adaptability to evolving technologies and changing user requirements. 

By focusing on these next steps, the service can be further refined to provide an even more informative, user-friendly, and effective tool for homeowners making sustainable energy decisions.


\section{Limitations}

This study encompasses certain limitations that need to be considered when interpreting the findings.
\begin{description}
    \item[Small sample size:] The study was conducted with a limited number of participants, which might affect the generalisability of the results.
    \item[Limited knowledge background:] Approximately half of the participants had a background in the energy domain, which could introduce bias into their perceptions and responses.
    \item[Restricted age range:] The age distribution of participants was concentrated around two main age groups: approximately 30 and 60 years old. This might limit the representation of perspectives across a wider age spectrum.
    \item[Lab studies] Due to time restrictions, evaluations were conducted in labs, constituting short-term studies. This might introduce bias and limit the ability to draw conclusions about the long-term performance of the artefact, necessitating more extended and quantitative studies.
    \item[Age-related technology challenges:] Participants aged 65 and above encountered challenges when engaging with the online tool. Older participants preferred face-to-face consultations and demonstrated higher trust in human experts than in an AI system, potentially affecting overall user experience and the willingness to adopt recommendations, especially among older demographics.
    \item[Restricted technology choices] Some technologies, like house wind energy generators, are not integrated into the system. The study acknowledges the need to consider integrating more technologies and updating them in a timely manner.
    \item[Neglect of rebound effect:] The study does not account for the rebound effect \cite{Herring2007}, which refers to potential changes in household behaviour that could result from adopting energy-efficient technologies and renewable energy. This omission could lead to an oversight in estimating the actual impact on energy consumption and subsequent energy bills. While the model assumes a certain comfort lifestyle as a baseline, any significant behavioural shifts induced by the recommended changes might not be accurately captured. Nevertheless, given the reference to comfort lifestyle, it is assumed that the rebound effect's influence would likely be minimal.
\end{description}
These limitations underscore areas for further investigation and potential refinement of the home energy system recommender to accommodate a broader range of users and contexts.