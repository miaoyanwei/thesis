\chapter{Conclusion}

According to Fogg's Persuasive design framework, 
German households exhibit strong "motivations" for adopting energy-efficient technologies. 
The proposed practice of a home energy system recommender works effectively as a "promot," as it provides valuable insights and recommendations to support their decision-making process. 
However, there is a lack of "ability" for many households, 
as they express frustration about the financial challenges of affording a one-time investment to purchase such technologies. 
They are hopeful that policy makers will take action to address this issue and provide support to make energy-efficient technologies more accessible and affordable.

For participants who did not think about invest into energy technologies,
they change their minds and 

Seems like the estimated inestment costs work more as a reference for people who have knowledge or interest in energy technologies. 
They don't take that estimates too seriously. 

For older participants, online tool is not very ideal to understand. 

All participants are aware of climate change and wish to do something to help, but this is not really a main factor for them to invest into expensive energy technologies. 

All participants mentioned they need goenrment supports since the initial investment of energy technologies requires a lot of money, current measure is not enough. 

\section{Limitations}

Half of the participants are having related knowledge. 
The age ranges of them are very limited, they are either around 30, or around 60. 

Lack of a consideration of the rebound effect \cite{Herring2007}. 
Doesn't consider if it will also lead to household's behaviour change and lead to miscalculation of the energy bills. 
However, the model takes the comfort lifestyle as a reference, so we believe the rebound effect would not affect too much. 
