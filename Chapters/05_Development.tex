\chapter{Development} 

The web application is designed as a client-server architecture, 
with the frontend responsible for collecting user data and presenting recommendations and explanations, 
while the backend handles the processing and analysis of data using the FLEX models.


\section{Frontend}

The frontend of the web application utilises HTML, CSS, and JavaScript to construct the user interfaces and manage the application's behaviour. 
HTML is employed to structure the webpage content and define its elements. 
CSS is used to style and format the appearance of the webpage. 
JavaScript, on the other hand, adds interactivity and dynamic functionality to the webpage, enabling features such as language switching and server communication. 
Furthermore, SurveyJS, an open-source JavaScript form builder library, is integrated into the frontend to provide a convenient solution for creating and embedding surveys within the application. 
In summary, HTML, CSS, JavaScript, and external tools like SurveyJS are combined in the frontend development to create visually appealing, interactive, and user-friendly interfaces. 


\section{Backtend}

Django, a Python-based web framework, is employed as the backend of the web application. 
This choice is driven by the fact that the FLEX models are written in Python. 
By using Django, seamless communication between the backend and the models is facilitated. 
The backend, built with Django, serves as the intermediary between the frontend and the FLEX models, allowing for efficient data processing and retrieval. 
