\chapter{Development} 

The web application is designed, 
with the frontend responsible for collecting user data and presenting recommendations and explanations, 
while the backend handles the database generated by the FLEX models. 


\section{Frontend}

The frontend of the web application is responsible for creating an engaging and user-friendly interface using HTML, CSS, and JavaScript. 
They were used to structure the content, define the visual styles, and add interactivity to the application.
HTML is used to create the structure of the webpages, CSS is employed to style the visual appearance of the application. 
JavaScript plays a crucial role in adding interactivity and dynamic functionality to the web application. 
Additionally, JavaScript is responsible for making asynchronous requests to the server, facilitating communication with the backend.
To ensure a responsive design, the web application utilises the Bootstrap framework. 
Although the service is not intended for mobile screens, a responsive user interface that adapts to various devices and screen sizes has been taken into account. 


\subsection{Questionnaire}

To incorporate questionnaires into the web application, we integrated SurveyJS, an open-source JavaScript form builder library \cite{surveyjs}. 
SurveyJS simplifies the process of creating and embedding surveys.
It supports logic and branching, allowing for dynamic survey behaviour based on user responses, 
that fulfils our need of presenting corresponding questions according to the answers, as decribed in the design section. 


\subsection{Charts}

For chart building, we initially opted for Google Charts, a charting library provided by Google \cite{googlecharts}. 
However, we encountered difficulties in building multiple columns using Google Charts. 
As a result, we switched to Highcharts \cite{highcharts}, another powerful charting library written in JavaScript. 


\section{Backend}

The backend of the web application utilises Flask, a Python-based web framework \cite{flask}, to serve as the intermediary between the frontend and the FLEX models. 
This choice was made based on the fact that the FLEX models are implemented in Python. 
Originally, our intention was to enable direct communication between the backend and the models using Python. 
However, during the development process, we realised that the models' calculations, especially when finding recommended configurations, could be time-consuming. 
Each scenario takes approximately seven seconds to calculate, and considering the need to identify multiple scenarios that could save energy costs for the household, 
it would be impractical to make the user wait for the results. 
To address this issue, we decided to pre-process the data in the FLEX models and store it in a database. 
This approach significantly reduced the time required to identify energy-saving scenarios, allowing for a more efficient user experience.


\subsection{JSON Schema Documentation}

The API design for the service follows a RESTful architecture and adheres to the JSON schema presented in this section. 
This JSON schema defines the structure and properties of a household's energy system and recommendation.


\subsubsection{Household's energy system and recommendation}


\subsubsection{Properties}

The documentation consists of three main components: profile, current, and recommendation as displayed in table \ref{tab:properties}. 

\begin{table}[h!]
    \centering
    \small
    \begin{tabular}{ | p{.10\textwidth} | p{.10\textwidth} | p{.35\textwidth} | p{.10\textwidth} | p{.10\textwidth} | } 
      \hline
      Name & Type & Description & Required & Default value \\
      \hline
      profile & object & An object describing the house's location and number of people. & Yes & - \\
      \hline
      current & object & An object describing the house's current energy system configurations, energy data, and costs. & Yes & - \\
      \hline
      recom-men-dation & array & A list of recommended configurations that improve the house's energy efficiency. & Yes & - \\
      \hline
    \end{tabular}
    \caption{Properties}
    \label{tab:properties}
\end{table}


\subsubsection{Properties of profile}

As table \ref{tab:properties_profile} shows, the profile component provides information about the house's location and the number of people residing in it. 
It includes properties such as location and person.

\begin{table}[h!]
    \centering
    \small
    \begin{tabular}{ | p{.10\textwidth} | p{.10\textwidth} | p{.35\textwidth} | p{.10\textwidth} | p{.10\textwidth} | } 
    \hline
    Name & Type & Description & Required & Default value \\
    \hline
    location & string & The location of the house. & Yes & - \\
    \hline
    person & integer & The total number of people residing in the house. & Yes & - \\
    \hline
    \end{tabular}
    \caption{Properties of profile}
    \label{tab:properties_profile}
\end{table}


\subsubsection{Properties of current}

The current component describes the house's current energy system configurations, energy data, and costs. 
It consists of two properties: config and energy\_data. 
See table \ref{tab:properties_current}.

\begin{table}[h!]
    \centering
    \small
    \begin{tabular}{ | p{.10\textwidth} | p{.10\textwidth} | p{.35\textwidth} | p{.10\textwidth} | p{.10\textwidth} | } 
    \hline
    Name & Type & Description & Required & Default value \\
    \hline
    config & object & An object describing the house's current energy system configurations. & Yes & - \\
    \hline
    \makecell{energy\_\\data} & object & An object describing the energy demand, PV generation, and energy cost. & Yes & - \\
    \hline
    \end{tabular}
    \caption{Properties of current}
    \label{tab:properties_current}
\end{table}


\subsubsection{Properties of config}

The config property, as shown in table \ref{tab:properties_config}, captures the current energy system configurations, including parameters such as pv\_size, battery\_capacity, sems, heating\_system, heating\_system\_type, and building\_renovation.

\begin{table}[h!]
    \centering
    \small
    \begin{tabular}{ | p{.10\textwidth} | p{.10\textwidth} | p{.35\textwidth} | p{.10\textwidth} | p{.10\textwidth} | } 
    \hline
    Name & Type & Description & Required & Default value \\
    \hline
    pv\_size & integer & Determine the size of the PV system. & Yes & - \\
    \hline
    \makecell{battery\_\\capacity} & integer & Determine the capacity of the battery system. & Yes & - \\
    \hline
    sems & boolean & Determine the state of a SEMS system. & Yes & - \\
    \hline
    \makecell{heating\_\\system} & boolean & Determine the state of the heating system used.	 & Yes & - \\
    \hline
    \makecell{boiler\_\\type} & string & Determine the type of heating system used. & Yes & - \\
    \hline
    \makecell{building\_\\renovation} & boolean & Determine the state of the renovation. & Yes & - \\
    \hline
    \end{tabular}
    \caption{Properties of config}
    \label{tab:properties_config}
\end{table}


\subsubsection{Properties of energy\_data}

The energy\_data property contains data related to energy demand, PV generation, and energy cost. 
It includes properties like energy\_demand, energy\_generate, heating, cooling, appliance, hotwater, pv, and energy\_bill\_year.
See table \ref{tab:properties_energydata}. 

\begin{table}[h!]
    \centering
    \small
    \begin{tabular}{ | p{.10\textwidth} | p{.10\textwidth} | p{.35\textwidth} | p{.10\textwidth} | p{.10\textwidth} | } 
    \hline
    Name & Type & Description & Required & Properties in Database \\
    \hline
    \makecell{energy\_\\demand} & integer & The total energy demand in a year. & Yes & - \\
    \hline
    \makecell{energy\_\\generate} & integer & The total energy generated by PV in a year. & Yes & - \\
    \hline
    heating & array & The energy demanded for heating in the house for each month. & Yes & \makecell{E\_\\Heating\_\\HP\_out \\+ Q\_\\Heating\\Element} \\
    \hline
    cooling & array & The energy demanded for cooling in the house for each month. & Yes & \makecell{E\_\\Room\\Cooling} \\
    \hline
    appliance & array & The energy demanded by all appliances in the house for each month. & Yes & \makecell{BaseLoad\\Profile} \\
    \hline
    hotwater & array & The energy demanded for hot water in the house for each month. & Yes & \makecell{E\_DHW\_\\HP\_out} \\
    \hline
    pv & string & The energy generated from PV in the house for each month. & Yes & \makecell{Photo\\voltaic\\Profile} \\
    \hline
    \makecell{energy\_\\bill\_\\year} & integer & The total yearly energy cost. & Yes & - \\
    \hline
    \end{tabular}
    \caption{Properties of energy\_data}
    \label{tab:properties_energydata}
\end{table}


\subsubsection{Properties of recommendation}

The recommendation component represents a list of recommended configurations that can improve the house's energy efficiency. 
As listed in table \ref{tab:properties_recommendation}, each recommendation includes properties similar to the config and energy\_data properties in the current component. 
Additionally, it includes an investment\_cost property indicating the annualised investment cost for the recommended configuration. 

\begin{table}[h!]
    \centering
    \small
    \begin{tabular}{ | p{.10\textwidth} | p{.10\textwidth} | p{.35\textwidth} | p{.10\textwidth} | p{.10\textwidth} | } 
    \hline
    Name & Type & Description & Required & Default value \\
    \hline
    config & object & An object describing the recommended energy system configurations. & Yes & - \\
    \hline
    \makecell{energy\_\\data} & object & An object describing the energy demand, PV generation, and energy cost. & Yes & - \\
    \hline
    \makecell{investm\\ent\_cost} & integer & The annualised investment cost for the recommended configuration. & Yes & - \\
    \hline
    \end{tabular}
    \caption{Properties of recommendation}
    \label{tab:properties_recommendation}
\end{table}


\subsubsection{Example JSON data}

\begin{verbatim}
"energy_data": {
    "total_generate": <int>,
    "total_demand": <int>,
    "boiler": <int[12]>,
    "cooling": <int[12]>,
    "appliance": <int[12]>,
    "hotwater": <int[12]>,
    "pv": <int[12]>
},
\end{verbatim}

The JSON data example can be found in the appedix \ref{appendix:example_JSON}. 

\subsection{Endpoints}

The API exposes five endpoints to retrieve data,
they are survey\_scenario, scenario, recommendation, energy\_data, and eenergy\_cost.
The endpoints accept HTTP GET requests and returns JSON responses. 