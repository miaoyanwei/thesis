\chapter{The newTRENDs Project} 

The aim of NewTRENDs is to increase the qualitative and quantitative understanding of impacts of New Societal Trends on energy consumption and to improve the modelling of energy demand, energy efficiency and policy instruments \cite{fraunhofer}. 

\section{Concept}

%The newTRENDs project develops the analytical basis for a ``2050 Energy Efficiency Vision" by considering New Societal Trends in energy demand modeling \cite{newtrends}.

With increasing renewable generation integrated into the power system, supply-side fluctuations must be balanced by demand-side flexibility. 
Electrification and demand response (DR) are becoming increasingly relevant to the heating transition of buildings, which demands the diffusion of

\begin{itemize}
  \item heat pumps (HPs),
  \item photovoltaic (PV) and energy storage (e.g., battery and hot water tank),
  \item smart energy management systems (SEMSs).
\end{itemize}

Combining the three technologies is also beneficial from an individual household (or building) perspective. The household can optimize the heat pump operation to reduce energy costs by saving energy in the tanks or pre-heat the building when the electricity price is lower. Besides, the energy-saving benefit could be further increased with PV and battery system.
From a market perspective, DR flexibility and PV generation also facilitate the concept of "energy community". The households can trade electricity with each other (peer-to-peer, P2P) within a local micro-grid or even trade with the other side of the country through the national grid, depending on the infrastructure, business model, and policies. In addition, households can also buy the services from an "aggregator", who bundles and manages the flexibility of small consumers and producers and participate in the market activities (Kerscher and Arboleya 2022). 
Promoted by the declining costs of technologies and support policies, more and more household "consumers" are expected to become "prosumers" (with PV) and "prosumagers" (plus energy storage and SEMS) (Fereidoon Sioshansi 2019).

\section{FLEX models}

The FLEX-Operation and FLEX-Community models were built to improve the building modeling suite and to analyze the societal trends of prosumaging and energy communities.

\subsection{FLEX-Operation}
FLEX-Operation calculates the energy consumption of each representative building, including operation of technologies (e.g., battery, PV, heat pump, etc.) and load profiles in hourly resolution.

\subsection{FLEX-Community}
FLEX-Community can be applied to support the aggregators designing and evaluating business models, as well as making investment decisions, for example, the self-owned battery, PV panels, etc.

\section{Motivations}
\section{Research questions}

This paper focuses on ...

