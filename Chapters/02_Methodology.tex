\chapter{Methodology} 

The study uses Design Case Studies \cite{dcs} as the research framework. 


\section{Context Study}

In the pre-study phase, 
our main focus was to identify existing practices and tools that can help homeowners understand renewable energy and energy-efficient technologies.
We conducted extensive searches online and reviewed relevant literature to gather information. 
However, we encountered a challenge as there were limited successful initiatives available to guide homeowners effectively.
Despite this limitation, we identified two related options: energy audits and research models. 
To gain a deeper understanding of these options, we conducted an in-depth analysis of both energy audits and research models. 
The objective was to explore how these methods could potentially assist households in learning about the functionalities and benefits of various energy technologies.
Overall, we aimed to identify effective methods that can support homeowners in understanding the functionalities and potential benefits they offer.


\section{Design study}

After the pre-study, we came up with an innovative design - a home energy system recommender. 
To ensure its effectiveness, we first investigated the motivations of house owners when considering energy technology investments. 
Discovering that financial aspects were the main focus for our target users, we decided to use financial factors as the prompt to nudge them towards making sustainable choices.

We then implemented three recommending rules based on this information. 
To create a user-friendly experience, we simplified the hundreds of input values of the FLEX models, making it less confusing and upsetting for users. 
Similarly, we presented the output data in a simplified manner, using bar chart visualisations to display energy demand and generation information.

For a trusted recommendation system, we provided different perspectives of explanations and varied levels of explainability to foster trust among users.
Additionally, we offered other methods for users to access more information and gain confidence in the recommendations.

The interfaces were developed using Figma, and we conducted usability testing among HCI fellows after building the high-fidelity wireframes. 
Based on the feedback received from experts, we iterated on the design to improve its usability and experience.


\section{Appropriation Study}

After developing the service, 
we conducted an investigation into the appropriation of the technical artifact, carrying out qualitative evaluations with \textcolor{cyan}{N} participants. 
The testing process involved a task and a semi-structured interview, which took one to two hours per participant.

During the interviews, we sought to understand two main perspectives from the users. 
The first perspective was whether the users felt they had learned more about energy technologies through the service. 
The second perspective focused on the users' trust in the recommendations provided by the system.

After collecting the data from the interviews, we performed a thematic analysis to gain deeper insights into the users' experiences and perceptions. 


%The methodology adopted in this study is based on the Design Case Studies framework. 
%The pre-study phase will begin by conducting a comprehensive review of the literature to identify best practices for providing households with personalised and professional home energy system recommendations, as well as techno-economic assessments.
%Based on the findings from the pre-study, I will then design the interfaces of the intervention. 
%The interfaces will be developed to provide an intuitive and user-friendly experience that can easily be understood by households. 
%Following the development phase, real users will be invited to use the intervention, and feedback will be collected both qualitatively and quantitatively. 
%The qualitative data will be collected through interviews with participants, while the quantitative data will be collected through surveys. 
%Finally, the collected data will be analysed to evaluate households thoughts about the recommendations and energy technologies. 
%The entire process will be documented and reported in the form of a thesis. 
