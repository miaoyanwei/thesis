\chapter{Methodology} 

The study uses Design Case Studies \cite{dcs} as the research framework. 


\section{Literature review}

A comprehensive review of the literature aims to gain a thorough understanding of the current state of energy policies, 
as well as the importance of promoting energy efficiency at the household level.


\section{Secondary research}

Secondary research is then conducted to explore the motivators that influence households' investment decisions regarding energy technologies.
This research involves analysing existing surveys to identify key factors driving households to make investment choices related to energy technologies. 
Special attention is given to financial considerations. 

\section{Investigation of field applications}

The phase of the study concerning the investigation of field applications primarily focuses on exploring and examining practical strategies and approaches used to educate households about energy technologies. 
The objective is to identify effective methods that can be employed to assist households in making informed decisions regarding energy-efficient technologies. 
However, given the limited availability of successful initiatives in this area, alternative options such as energy audits and academic models are being explored.

%The methodology adopted in this study is based on the Design Case Studies framework. 
%The pre-study phase will begin by conducting a comprehensive review of the literature to identify best practices for providing households with personalised and professional home energy system recommendations, as well as techno-economic assessments.
%Based on the findings from the pre-study, I will then design the interfaces of the intervention. 
%The interfaces will be developed to provide an intuitive and user-friendly experience that can easily be understood by households. 
%Following the development phase, real users will be invited to use the intervention, and feedback will be collected both qualitatively and quantitatively. 
%The qualitative data will be collected through interviews with participants, while the quantitative data will be collected through surveys. 
%Finally, the collected data will be analysed to evaluate households thoughts about the recommendations and energy technologies. 
%The entire process will be documented and reported in the form of a thesis. 
