\chapter{Methodology} 

The study uses Design Case Studies \cite{dcs} as the research framework. 


\section{Context Study}

In the pre-study phase, 
our main focus was to identify existing practices and tools that can help homeowners understand renewable energy and energy-efficient technologies.
We conducted extensive searches online and reviewed relevant literature to gather information. 
However, we encountered a challenge as there were limited successful initiatives available to guide homeowners effectively.
Despite this limitation, we identified two related options: energy audits and research models. 
To gain a deeper understanding of these options, we conducted an in-depth analysis of both energy audits and research models. 
The objective was to explore how these methods could potentially assist households in learning about the functionalities and benefits of various energy technologies.
Overall, we aimed to identify effective methods that can support homeowners in understanding the functionalities and potential benefits they offer.


\section{Design study}

comes up with an innovative design for an ICT artifact related to the findings of the first phase


\section{Appropriation Study}

investigates into the appropriation of the technical artifact over a longer period of time. 

%The methodology adopted in this study is based on the Design Case Studies framework. 
%The pre-study phase will begin by conducting a comprehensive review of the literature to identify best practices for providing households with personalised and professional home energy system recommendations, as well as techno-economic assessments.
%Based on the findings from the pre-study, I will then design the interfaces of the intervention. 
%The interfaces will be developed to provide an intuitive and user-friendly experience that can easily be understood by households. 
%Following the development phase, real users will be invited to use the intervention, and feedback will be collected both qualitatively and quantitatively. 
%The qualitative data will be collected through interviews with participants, while the quantitative data will be collected through surveys. 
%Finally, the collected data will be analysed to evaluate households thoughts about the recommendations and energy technologies. 
%The entire process will be documented and reported in the form of a thesis. 
