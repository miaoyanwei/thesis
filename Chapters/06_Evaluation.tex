\chapter{Evaluation}

In this chapter, we present the evaluation of our IT artefact. 
The evaluation study focuses on two main aspects. 
Firstly, we assess the effectiveness of the explanations provided to users in helping them understand why specific configurations were recommended for their home energy systems.
Secondly, we evaluate whether the information presented through the explanations has an impact on users' attitudes, such as increased awareness of energy-efficient technologies or a stronger inclination to adopt them. 
 
As highlighted in Nunes and Jannach's summary \cite{Nunes2020}, 
there is no universally accepted definition of what constitutes a correct or best explanation, 
evaluating the quality of explanations relies on capturing the subjective perceptions of users and monitoring the impact of these explanations on user behaviour, 
and user studies have been the predominant research method for assessing explanations in recommender systems.
Aligned with this understanding, our goal of capturing the subjective perceptions of users regarding the explanations provided and assessing any changes in their attitudes, 
we have made the decision to conduct real-user studies. 

Due to time constraints imposed by the university's requirements for a master's thesis, we conducted qualitative user studies with \textcolor{cyan}{6} participants in the \textcolor{cyan}{South Germany} region. 
While the sample size was small, qualitative studies provide valuable insights into rusers' perceptions, attitudes, and experiences.


\section{Semi-structured interviews}

The interview will follow the following structured format to gather insights from the participants. 
\begin{enumerate}
  \item Express gratitude to the interviewee for their willingness to participate and introduce myself briefly.
  \item Provide a brief recap of the project, emphasising its purposes, and clarify the specific objectives of the interview.
  \item Inform the interviewee about the expected duration of the interview (approximately 40 minutes) and assure them that the recording will be used solely for transcription purposes, ensuring confidentiality.
  \item Explain that the interview will involve participants using the web service to explore and discover recommendations tailored to their home energy systems.
  \item Prior to participants using the service, they will be asked a series of questions pertaining to their demographic information and initial perceptions of energy technologies.
  \item Allow participants to utilise the service using a laptop.
  \item Once the participants have interacted with the service, other questions will be asked.
  \item Inquire if the interviewee has any remaining questions or uncertainties.
  \item Express appreciation once again for their participation and inform them that they can reach out with any further concerns or inquiries they may have.
\end{enumerate}


\subsubsection{Target groups}

The target users of this study are individuals residing in Germany who own single-family houses.
\begin{table}[h!]
  \centering
  \begin{tabular}{ | p{.10\textwidth} | p{.10\textwidth} | p{.10\textwidth} | p{.20\textwidth} | p{.20\textwidth} | } 
    \hline
    ID & Gender & Age & Location & Background \\
    \hline
    P1 & M & 60 & Germany & Construction \\
    \hline
  \end{tabular}
  \caption{Participants}
  \label{tab:participants}
\end{table}


\subsubsection{Goals}

\begin{itemize}
  \item Evaluate the effectiveness of explanations in helping users understand the rationale behind recommended configurations for their home energy systems.
  \item Assess the impact of the provided information on users' attitudes, including increased awareness of energy-efficient technologies and a higher propensity to adopt them. 
\end{itemize}


\subsubsection{Material}

\begin{itemize}
  \item \textbf{Laptop:} Participants will be provided with a laptop to access and use the web service before the interview. 
  \item \textbf{Recording Device:} A mobile phone for instance, will be used to capture and record the interview session to enable accurate transcription of the interview responses for analysis and reference.
  \item \textbf{Interview Guideline:} A printed copy of the interview guideline.
  \item \textbf{Pen and Papers:} A pen and some papers to jot down any notes or additional information during the session. 
  \item \textbf{Translator:} In the event that participants are not comfortable with the English language, a German speaker will be present to assist in facilitating communication and ensuring a clear understanding of the questions and responses.
\end{itemize}


\subsubsection{Welcome}

Thank you for participating in this interview. 
My name is Yanwei Miao, and I am currently working on my master's thesis project in collaboration with Fraunhofer ISI.
We developed a web service to help homeowners like you make informed decisions about investing in energy technologies for your homes. 
Our aim is to provide personalised recommendations based on your specific circumstances, enabling you to determine the economic feasibility of implementing these technologies. 
To proceed, we kindly request your cooperation in answering a few questions about your background. 
This will help us gain a better understanding of your knowledge and perspective on AI and energy. 
Once is complete, we will guide you through the web service to obtain personalised recommendations for your home energy system. 
While using the service, we encourage you to think out loud and share your thoughts and observations. 
Afterward, we will conduct a 30-minute interview to gather your feedback and insights. 
If you have any questions at any point, please don't hesitate to ask. 
Are you ready to begin?


\subsubsection{Questions}

The interview questions can be categorised into four main categories: \emph{demography, explainability, attitude change, and additional questions}.
Demography questions focus on gathering information about the background of the interviewees, aiming to identify any demographic factors that may influence their interest in more detailed explanations.
Explainability questions are designed to assess the clarity and comprehensibility of the provided explanations, aiming to determine if they are clear and understandable to the participants. 
Attitude change questions are divided into two parts: before using the service and after using the service. 
These questions aim to capture any changes in participants' attitudes towards energy technologies and their perception of the recommendations after using the service.
Lastly, additional questions or thoughts may arise during the interview. 
They could be an opportunity to explore additional insights or address any specific concerns during the conversation. 

\begin{center}
  \small
  \begin{longtable}{ | p{.20\textwidth} | p{.70\textwidth} | }
    \hline
    \textbf{Category} & \textbf{Questions} \\
    \hline
    \multicolumn{2}{|l|}{\cellcolor{lightgray}Before using the web service} \\
    \hline
    \multirow{6}{4em}{Demography} & Gender \\
    & Age \\
    & Educational background \\
    & Occupation \\
    & Knowledge and interest in AI \\
    & Knowledge and interest in the energy domain \\
    \hline
    \multirow{4}{4em}{Altitudes} & Have you heard of energy-efficient appliances or renewable energy technologies for households? \\
    & Have you ever considered implementing energy-efficient technologies, such as solar panels and smart thermostats in your house? \\
    & What is your understanding regarding the benefits of energy-efficient technologies? \\
    & Do you know climate change and why it is important for individuals to save energy and utilise renewable energy sources? \\
    \hline
    \multicolumn{2}{|l|}{\cellcolor{lightgray}After using the web service} \\
    \hline
    \multirow{4}{4em}{Altitudes} & How do you feel about the recommendations provided? \\
    & Do you find the recommendations useful or valuable? \\
    & Are you considering investing in any of the recommended technologies now? Why or why not? \\
    & What factors influence your decision to adopt or reject the recommendations? \\
    \hline
    \multirow{7}{4em}{Explainability} & Do you know why the recommendations were recommended to you? \\
    & Do you trust the recommendations? Why or why not? \\
    & What factors contribute to your trust or lack of trust in the recommendations? \\
    & Were you familiar with these technologies before using the system? \\
    & Did the system provide enough information for you to understand the technologies? \\
    & Has your knowledge of energy efficient technologies improved as a result of using the system? \\
    & Do you believe adopting these technologies can lead to lower energy costs? Why or why not? \\
    \hline
    \multirow{2}{4em}{Additionals} & Give participants an opportunity to share any additional thoughts, concerns, or suggestions regarding the system and its recommendations. \\
    & Ask if they have any questions for you or if there's anything else they would like to discuss. \\
    \hline
  \caption{Interview guideline}
  \label{tab:interview}
  \end{longtable}
\end{center}


\subsubsection{Closure}

Thank you so much for taking the time to participate in this interview session. 
I truly appreciate your willingness to share your thoughts and experiences with us. 
If you have any further questions, concerns, or additional insights that you would like to share, please don't hesitate to reach out.


\section{Future work: Kano survey}

The Kano model \cite{Sauerwein1996} is a commonly used framework in quantitative research to understand customer satisfaction and prioritise features or attributes. 
In our project, we aim to also incorporate the Kano model survey (Table \ref{tab:kanomodel}) as part of our future work. 
We plan to integrate this survey directly into the web service, allowing individuals who have interacted with the service online to voluntarily complete the survey.
Through the integration of the Kano survey into the web service, we have the opportunity to collect insights from users who engage with the service online. 
This will enable us to assess whether the inclusion of specific features in the service brings delight to our users.
This assessment will help us determine the necessity of explanations in the service. 

\begin{center}
  \small
  \begin{longtable}{ | p{.40\textwidth} | p{.06\textwidth} | p{.06\textwidth} | p{.06\textwidth} | p{.06\textwidth} | p{.06\textwidth} |}
    \hline
    Features & I like it & I expect it & I'm neutrual & I can tolerate it & I dislike it \\
    \hline
    Show corresponding yearly energy bill &&&&& \\
    \hline
    Don't show corresponding yearly energy bill &&&&& \\
    \hline
    Show detailed simulated yearly energy consumption &&&&& \\
    \hline
    Don't show detailed simulated yearly energy consumption &&&&& \\
    \hline
    Show detailed simulated daily energy consumption &&&&& \\
    \hline
    Don't show detailed simulated daily energy consumption &&&&& \\
    \hline
    Show climate change information &&&&& \\
    \hline
    Don't show climate change information &&&&& \\
    \hline
    Allow exploring and adjusting configurations of the recommended technologies &&&&& \\
    \hline
    Don't allow exploring and adjusting configurations of the recommended technologies &&&&& \\
    \hline
    Show comparison with current situation &&&&& \\
    \hline
    Don't show comparison with current situation &&&&& \\
    \hline
    Show explanation of each technology &&&&& \\
    \hline
    Don't show explanation of each technology &&&&& \\
    \hline
    Show total investment costs of each technology &&&&& \\
    \hline
    Don't show total investment costs of each technology &&&&& \\
    \hline
    Show annualised investment costs of each technology &&&&& \\
    \hline
    Don't show annualised investment costs of each technology &&&&& \\
    \hline
  \caption{Kano survey}
  \label{tab:kanomodel}
  \end{longtable}
\end{center}


\section{Analysis}

Table \ref{tab:participants_perceptions} summarises the participants' overall knowledge and interest in AI and energy, 
as well as their perspective on the altitude change and trust in the recommendations following their use of the service.  

\begin{table}[h!]
  \centering
  \begin{tabular}{ | p{.10\textwidth} | p{.15\textwidth} | p{.15\textwidth} | p{.15\textwidth} | p{.15\textwidth} | } 
    \hline
    ID & Interest in AI & Interest in energy & Altitude change & Trust \\
    \hline
    P1 & No & Yes & No & Yes \\
    \hline
  \end{tabular}
  \caption{Participants' perceptions about AI and energy}
  \label{tab:participants_perceptions}
\end{table}


\subsubsection{Participant 1}

Before using the service, 
Participant 1, who works in the house construction industry, 
displayed knowledge of EU energy policy and familiarity with energy-efficient appliances and renewable energy technologies. 
He had implemented solar panels in one of his houses and believed in their ability to reduce energy costs for households. 
Participant 1 emphasised the rising energy prices and the importance of addressing climate change, expressing dissatisfaction with Germany's continued use of coal. 

During the service usage, 
Participant 1 found the service to work smoothly, receiving a comprehensive list of around 50 recommendations to lower energy costs. 
While he agreed with most configurations, he felt overwhelmed by the extensive list and preferred to focus on specific options such as installing a heat pump and conducting a house renovation, given his expertise in his own house's needs. 
Despite not finding recommendations for biomass utilisation, Participant 1 relied on his professional experience and personal resources in considering biomass as the most cost-effective option. 
Although he missed the last level of explanation on renewable energy benefits, he clicked to view more details on the specific recommendation he sought.

After using the service, 
Participant 1 appreciated the recommendations but felt they were more suited for general houses rather than his unique situation. 
Financial considerations and environmental impact guided his energy technology decisions. 
Trusting the recommendations due to his familiarity with them, he acknowledged the service provided an overwhelming amount of information and a surplus of recommendations.