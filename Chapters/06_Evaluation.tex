\chapter{Evaluation}

In this chapter, we present the evaluation of our IT artefact. 
The evaluation study focuses on two main aspects. 
Firstly, we assess the effectiveness of the explanations provided to users in helping them understand why specific configurations were recommended for their home energy systems.
Secondly, we evaluate whether the information presented through the explanations has an impact on users' attitudes, such as increased awareness of energy-efficient technologies or a stronger inclination to adopt them. 
 
As highlighted in Nunes and Jannach's summary \cite{Nunes2020}, 
there is no universally accepted definition of what constitutes a correct or best explanation. 
Evaluating the quality of explanations relies on capturing the subjective perceptions of users and monitoring the impact of these explanations on user behaviour, 
and user studies have been the predominant research method for assessing explanations in recommender systems.
In line with this understanding, our aim is to capture the subjective perceptions of users regarding the explanations provided. To achieve this, we have chosen to conduct real-user studies. 

Due to time constraints imposed by the university's requirements for a master's thesis, we conducted qualitative user studies with \textcolor{cyan}{a limited number of} participants in the \textcolor{cyan}{South Germany} region. 
While the sample size was small, qualitative studies provide valuable insights into rusers' perceptions, attitudes, and experiences.


\section{Target groups}



\section{Semi-structured interviews}

\begin{center}
    \small
    \begin{longtable}{ | p{.20\textwidth} | p{.70\textwidth} | }
            \hline  
            \textbf{Category} & \textbf{Questions} \\
            \hline
            \multirow{4}{4em}{Demography} & Educational background \\
            & Occupation \\
            & Knowledge and interest in AI \\
            & Knowledge and interest in the energy domain \\
            \hline
            \multirow{7}{4em}{Explainability} & Do you know why the recommendations were recommended to you? \\
            & Do you trust the recommendations? Why or why not? \\
            & What factors contribute to your trust or lack of trust in the recommendations? \\
            & Were you familiar with these technologies before using the system? \\
            & Did the system provide enough information for you to understand the technologies? \\
            & Has your knowledge of EE technologies improved as a result of using the system? \\
            & Do you believe adopting these technologies can lead to lower energy costs? Why or why not? \\
            \hline
            \multirow{4}{4em}{Altitudes} & How do you feel about the recommendations provided? \\
            & Do you find the recommendations useful or valuable? \\
            & Are you considering investing in any of the recommended technologies? Why or why not? \\
            & What factors influence your decision to adopt or reject the recommendations? \\
            \hline
            \multirow{2}{4em}{Additionals} & Give participants an opportunity to share any additional thoughts, concerns, or suggestions regarding the system and its recommendations. \\
            & Ask if they have any questions for you or if there's anything else they would like to discuss. \\
            \hline
    \caption{Interview guideline}
    \label{tab:interview}
    \end{longtable}
  \end{center}


\section{Results}
