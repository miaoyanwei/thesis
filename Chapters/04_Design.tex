\chapter{Design} 

Introductory lines...

\section{Understanding current home energy system}


\subsection{Household profiles}

The concept of household profile has been developed to provide a comprehensive understanding of the energy consumption patterns of residential buildings. 
This approach takes into account a range of factors, including the external environment, building materials, energy consumption behaviors, and home energy systems. 
The aim of creating such a profile is to gain insights into the energy demand and supply dynamics of households. 
This comprehensive analysis also offers insights into the specific factors that contribute to household energy consumption and highlights potential opportunities for tailored improvements to home energy systems. 


\subsection{Decision trees for asking questions}

A total of 18 questions were raised to collect all the relevant information necessary for the household profile analysis. 
In order to optimize the user experience, a decision tree approach was employed, allowing users to navigate through the questionnaire without the need to answer all the questions. 


\section{Recommending improvements for home energy system}

\subsection{Recommendations}

An effective home energy system should prioritize minimizing energy waste, reducing dependence on non-renewable fossil fuels, and lowering overall energy costs. Our recommendations are aligned with these fundamental principles and aim to promote sustainable energy practices while also reducing household energy expenditures. 

The objectives of the recommendation system are multi-fold. 
Firstly, the system aims to support homeowners in making informed decisions regarding investments in home energy systems. 
Additionally, the system intends to encourage behavior change among homeowners by promoting the utilization of renewable energy sources. 
Finally, the recommendation system seeks to continuously refine and improve the accuracy of its predictive model, ensuring that the recommendations provided are up-to-date and effective. 
By providing users with tailored recommendations, the system aims to facilitate the adoption of energy technologies, ultimately leading to reduced energy demand and associated costs. 

As noted by Karen Palmer et al. \cite{informationgap}, financial considerations are of primary importance to homeowners when making decisions about energy investments. 
In line with this understanding, the recommendation system places a strong emphasis on providing transparent cost estimates for energy bills could be saved as well as recommended home energy system configurations. 
Additionally, the system seeks to encourage behavior change by providing information and education on climate change and renewable energy sources, aimed at increasing user awareness and understanding of the benefits of sustainable energy practices. 
To facilitate ongoing improvement and refinement of the recommendation system, a feedback survey button will be incorporated, allowing users to provide both short-term and long-term feedback on the system's performance and recommendations. 

\subsection{Explainability}

In order to provide more comprehensive and understandable recommendations, we have chosen to explain our recommendations from multiple perspectives beyond just cost estimates. 
Specifically, we have identified user-perceived quality factors, including trust, effectiveness, education, and debugging, as key aspects to incorporate into our explanations. 
The overall purposes of providing such explanations are multifaceted. 
First, transparency is essential to provide accountability in the decision-making process, particularly in situations where users may have doubts or reservations about the recommendations. 
Second, building trust and confidence in the recommendation system is critical to ensure user adoption and acceptance. 
Third, explanations can aid in user understanding, particularly when the recommended item is not immediately intuitive or apparent. 
Fourth, providing users with greater control over the recommendation process can help ensure that recommendations are aligned with their goals and preferences. 
Finally, facilitating user learning and exploration can lead to the discovery of new items or preferences that users may not have previously considered. 
By incorporating these quality factors into our explanations, we aim to provide recommendations that are transparent, trustworthy, understandable, and user-centric. 

Our recommendation system employs a three-level explainability framework to enhance user understanding of household energy consumption and the recommended home energy system configurations. 
At the first level, the system provides an end-result explanation in terms of the expected energy bill for the household. 
At the second level, the system offers a behavioral explanation of energy consumption patterns and the factors driving them. 
Finally, at the third level, the system aims to increase users' awareness and understanding of renewable energy and environmental protection.

Furthermore, the explanation process is divided into three layers: descriptive, diagnostic, and counterfactual. 
At the descriptive layer, users are provided with a comprehensive summary of their current energy consumption patterns. 
At the diagnostic layer, users are introduced to the various functionalities and benefits of the recommended energy technologies, including cost-saving potential and environmental impact. 
Finally, at the counterfactual layer, users are presented with simulated energy consumption data, allowing them to see the potential energy savings that could result from adopting the recommended configurations. 
By employing this multifaceted approach to explainability, we aim to provide users with a clear and nuanced understanding of their energy consumption habits and the potential benefits of transitioning to more sustainable energy systems. 


\section{Interactions}

\subsection{Interfaces}

To facilitate user understanding of the recommended home energy system configurations and associated costs, our recommendation system will employ a visual and natural language explanation interface. 
Specifically, an interactive visualization tool will be implemented to enable users to explore and compare different energy system configurations in terms of energy consumption patterns and costs. 
Additionally, natural language explanations will be provided to further enhance user understanding and engagement with the recommended configurations. 


\subsection{Data visualisation}


