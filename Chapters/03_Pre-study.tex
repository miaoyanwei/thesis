\chapter{Pre-study} 

In 2023, Eurostat \cite{eurostat} released a report providing insight into the energy consumption patterns of households, with a focus on its various end-uses such as space and water heating, space cooling, cooking, lighting and electrical appliances, as well as other peripheral uses of energy. Notably, the report revealed that in 2020, the residential sector, or households, accounted for a significant proportion of final energy consumption in the European Union (EU), representing 27.4\% of total final energy consumption or 18.7\% of gross inland energy consumption. This highlights the importance of understanding household energy usage in devising effective energy management and conservation strategies within the EU. 


\section{Academia}

Within the academic community, 
there exists a concerted effort to address issues related to energy efficiency and the adoption of renewable energy sources. 
There are numerous research and ongoing projects aimed at establishing a robust social energy infrastructure capable of adapting to the utilization of renewable energy sources. 
Studies are also investigating the feasibility and practicalities of establishing zero-emission households and buildings.
Technological innovations designed to support these goals have also been a focal point in the academic discourse surrounding energy efficiency and the transition to renewable energy.

In the meantime, Palmer et al. \cite{informationgap} drew attention to the fact that engineering studies have identified various investments in new energy-efficient equipment or building retrofits that would generate savings surpassing their costs in terms of lower future energy expenses. However, homeowners and businesses lack sufficient knowledge and guidance on how to effectively utilize these opportunities to their advantage.

\section{Industry}


\section{Regulations}


