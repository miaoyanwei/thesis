\chapter{Introduction} 

\section{Background}

Human-induced climate change 
is causing dangerous and widespread disruption in nature 
and affecting the lives 
of billions of people around the world \cite{ipcc}. 
To tackle climate change and its negative impacts, 
the historic Paris Agreement 
sets long-term goals to guide all nations 
to substantially reduce global greenhouse gas (\gls{ghg}) emissions 
to limit the global temperature increase
to 2 degrees Celsius in this century \cite{paris}. 
On European Union (\gls{eu}) level, 
“Energy 2020. A strategy for competitive, sustainable and secure energy”, 
published in November 2010,
and “Energy Roadmap 2050”, 
published at the end of 2011,
are the most important strategy papers currently,  
pointing the direction 
for energy developments in the \gls{eu} \cite{roadmap}. 
The aim is to confirm Europe's commitment 
to lead in global climate action 
and to present a vision 
that can lead to achieving net-zero \gls{ghg} emissions by 2050 
through a socially-fair transition 
in a cost-efficient manner \cite{clean}. 

To achieve these goals, 
two central strategies are pursued 
by the \gls{eu} and its Member States 
concerning the energy system \cite{2050}: 

\begin{enumerate}
  \item Enhancing energy efficiency (EE). 
  \item Decarbonizing energy supply, 
  in particular via large diffusion 
  and wide-use of renewable energy sources.
\end{enumerate}

In 2019, 
80.9\% of our total energy supply still depended on burning fossil fuels, 
namely 
26.8\% coal, 30.9\% oil and 23.2\% natural gas \cite{iea}. 
Nonetheless, investments into low-carbon power generation 
accounted for 15\% recently 
are expected to rise to more than 30\% by 2030, 
corresponding to a quadrupling in absolute volumes \cite{shift}. 
Solar, wind, and the investments 
for enabling the integration of these technologies to the grid 
dominate the investments into low-carbon power generation \cite{shift}. 

In additon, 
previous research has shown that 
in many areas energy efficiency gains 
were counteracted by societal trends 
that increased corresponding activities, 
leading to much smaller decreases (or even increases) 
of energy demand 
than technologically feasible \cite{2050}. 
Therefore, 
it is important to 
access current and (foreseeable) future societal trends 
concerning the impact that they might have 
on future energy demand \cite{2050}.

Climate change can only be tackled 
if people actively engage, 
as consumers and as citizens \cite{clean}. 

\section{newTRENDs Project}

The newTRENDs project 
develops the analytical basis 
for a "2050 Energy Efficiency Vision" 
by considering New Societal Trends in energy demand modeling \cite{newtrends}.

\section{Motivations}

Some text here ...

