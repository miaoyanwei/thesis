\chapter{Introduction}


\section{Background}
Human-induced climate change is causing dangerous and widespread disruption in nature, thereby affecting billions of lives globally \cite{ipcc}. 
To tackle climate change and its negative impacts, two main strategies are addressed: climate change mitigation and adaptation.

\begin{itemize}
  \item \textbf{Climate change mitigation} refers to the actions taken to reduce or prevent greenhouse gas (\gls{ghg}) emissions and ultimately stabilize the concentration of these gases in the atmosphere to limit global warming and its adverse effects \cite{handbook}.
  This goal entails a range of related projects, spanning farming, land use, peatland management, renewable energies, and energy efficiency. Integrated projects that implement climate change mitigation strategies and action plans at regional or national levels are also pertinent \cite{ec}.
  Notably, to curb carbon dioxide (\gls{co2}) emissions in the energy system, two main approaches are pursued:
\emph{
  (1) reducing energy consumption on the demand side through efficiency improvement and behavioral changes and
  (2) transitioning to renewable energy sources on the supply side.
}
%These terms go hand-in-hand as we tackle the climate crisis.
  \item \textbf{Climate change adaptation} encompasses measures to manage the adverse impacts of climate change, such as natural disasters, changes in precipitation patterns, and rising sea levels, among others \cite{handbook},
  which includes projects relating to urban adaptation and land-use planning, infrastructure resilience, sustainable water management in drought-prone areas, flood and coastal management, as well as the resilience of the agricultural, forestry, and tourism sectors \cite{ec}.  
\end{itemize}

The work in this thesis belongs to the category of climate change mitigation. 
%Following the work in the newTRENDs project\footnote{https://newtrends2020.eu/}, 
%we look at how the impact of ``new societal trends'' on the future development of the energy demand.
%Then, we apply the HCI techniques in designing a tool to guide decisions on household's investments in energy efficiency and renewables from a techno-economic perspective.  %the basic motivation of this thesis in one sentence.
%Climate change can only be tackled if people actively engage, as consumers and as citizens \cite{clean}.


\subsection{Mitigating climate change through energy transition}
The Paris Agreement, a historic international agreement, sets long-term goals to substantially reduce global emissions and limit the global temperature increase to 2 degrees Celsius in this century \cite{paris}.
To achieve this ambitious goal, the world is facing an unprecedented imperative to a rapid transition in the energy sector. 
The European Union (\gls{eu})'s "Energy 2020. A strategy for competitive, sustainable and secure energy" and "Energy Roadmap 2050" are key strategy papers guiding energy developments in the \gls{eu} \cite{roadmap}, aiming to lead in global climate action and achieve net-zero emissions by 2050 through a socially-fair and cost-efficient transition \cite{clean}. 


\subsection{Households in energy transition}

Households are a crucial component of the energy transition, as they are responsible for a significant proportion of final energy consumption in the \gls{eu}, as highlighted by Eurostat's 2023 report. 
In fact, in 2020, the residential sector accounted for 27.4\% of total final energy consumption or 18.7\% of gross inland energy consumption in the \gls{eu} \cite{eurostat}. 
Therefore, reducing energy consumption in households through energy-efficient building construction and renovations, as well as digitalisation and smart demand-side management, can have a significant impact on achieving the \gls{eu}'s energy and climate targets \cite{building}. 
This underscores the importance of developing and implementing effective policies and strategies to promote energy efficiency and renewable energy use in households to facilitate the energy transition.


\subsection{Technologies for home energy system}

Technologies for home energy systems have rapidly advanced in recent years, with a growing focus on energy efficiency and renewable energy sources. 
Smart home technologies, such as energy management systems, allow households to optimise their energy consumption and reduce waste. 
Moreover, rooftop solar panels and home battery storage systems enable households to generate and store their own renewable energy, reducing dependence on the grid and lowering electricity bills. 
In addition, the integration of electric vehicles with home energy systems can further reduce household carbon emissions and provide a source of backup power. 
These technologies have the potential to significantly transform the way households consume and generate energy, contributing to a more sustainable and resilient energy system.


\section{Opportunity}

Despite the growing availability and accessibility of home energy technologies, there remains a significant information gap regarding their effective utilisation. 
A survey conducted by Palmer et al. \cite{informationgap} identified a lack of knowledge and guidance among homeowners, preventing them from maximising the benefits of these investments in terms of reducing future energy expenses. 
Therefore, there is an opportunity in exploring effective ways to educate and inform house owners on home energy technologies. 


\section{Research questions and aims}

The following research questions were raised throughout the process and guided the study: 
\begin{itemize}
  \item \textbf{What practice can effectively bridge the information gap for house owners in the adoption of renewable energy and energy-efficient technologies?}
\end{itemize}

\begin{itemize}
  \item \textbf{How to develop effective explanations that build trust for a recommender in supporting households making sustainable decisions? }
\end{itemize}

The aim of this study is to address the information gap and support house owners in their decision-making process regarding the adoption of clean energy and energy-efficient technologies. 
%The study also seeks to evaluate the effectiveness of such a nudging approach. 

%The following research objectives will aid in answering the research questions: 
%\begin{itemize}
%  \item Conduct a thorough review of relevant literature to identify existing gaps and opportunities in the field. 
%  \item Develop a user-friendly and accessible interface for providing information about energy-efficient technologies and their benefits to households.
%  \item Design and implement interactive tools that enable households to estimate the costs and benefits of adopting different energy-efficient technologies and renewable energy sources.
%  \item Conduct user studies to evaluate the impact of the developed interventions on households' thoughts of energy-efficient technologies and renewable energy sources, using both quantitative and qualitative methods.
%  \item Analyse the data collected from the user studies to gain insights into the users' needs and preferences, as well as the effectiveness of the interventions.
%  \item Write the thesis that presents the findings, conclusions and recommendations based on the research conducted. 
%  \item Investigating the data required by the FLEX-Operation model. 
%  \item Identifying the typical European household types and understanding their perceptions of the household energy system. 
%  \item Providing recommendations and evaluations of the technological performance and economic feasibility of household's energy system.
%  \item Designing the web application with user-centred approaches. 
%  \item Using data visualisation techniques to ensure explainability of the recommendations. 
%  \item Developing the frontend and backend web application. 
%  \item Evaluating the explainability of the recommendations from users' perspectives and measuring the impact of the households‘ perceptions towards proposed solutions. 
%  \item Allowing long-term event tracking for design iteration. 
%\end{itemize}

%Meanwhile, the following criteria should be taken into consideration 
%while building a software application from a user's perspective: 

%\begin{itemize}
%  \item The software should be easy and effortless to use. 
%  \item The interactions should be intuiative. 
%  \item The assessments should be clear explained to users. 
%\end{itemize}

%Accordingly, three subquestions are raised: 
%\begin{enumerate}
%  \item What are the data required by the FLEX-Operation model from households? 
%  \item What are the typical European household profiles? 
%  \item How to offer trustworthy and user-friendly recommendations to European households from a techno-economic perspective? 
%\end{enumerate}

This thesis seeks to contribute to the HCI community by offering insights into the design and development of personalised and professional home energy system recommendations and their effectiveness in promoting the adoption of those technologies. 
%The study will also provide insights into the user experience of the intervention and how it can be further improved to support sustainable energy choices. 
Overall, the findings of this thesis hopefully can inform the development of future HCI interventions to address environmental challenges. 


%\section{Justification and planning} 

%\subsection{Justification}

%The proposed thesis project will combine research and application aspects, 
%which is important because it will allow for a comprehensive understanding of the topic being studied. 
%Conducting a thorough literature review will provide a strong foundation for the research, 
%while the development of software applications will allow for practical implementation of the findings. 
%In addition, interviews with industry professionals and stakeholders will provide valuable insights into the real-world challenges and opportunities in the field. 
%Finally, the evaluation process will involve much data analysis, enabling the research to draw valid and reliable conclusions. 
%Thus, the proposed thesis project will contribute to a well-rounded and informative study, and is believed to be justified for 30 credits.


%\subsection{Supervision}

%This master thesis project will be supervised by 
%Prof. Dr. Gunnar Stevens (\href{mailto:gunnar.stevens@uni-siegen.de}{gunnar.stevens@uni-siegen.de}) at Siegen University and 
%Dr. Songming Yu (\href{mailto:songmin.yu@isi.fraunhofer.de}{songmin.yu@isi.fraunhofer.de}) from The Fraunhofer Institute for Systems and Innovation Research. 


%\subsection{Time planning}

%The following is the time allocation for the research objectives, which are scheduled to be completed in 26 weeks. \\

%\begin{table}[h]
%  \begin{center}
%    \begin{tabular}{ p{.05\textwidth} p{.80\textwidth} }
%      \cellcolor[rgb]{0.94,0.96,0.98}1 & Literature review \\
%      \cellcolor[rgb]{0.86,0.90,0.96}2 & Design interfaces\\ 
%      \cellcolor[rgb]{0.78,0.84,0.94}3 & Develope the software tool\\
%      \cellcolor[rgb]{0.71,0.78,0.92}4 & User studies \\
%      \cellcolor[rgb]{0.63,0.73,0.89}5 & Analyse data\\
%      \cellcolor[rgb]{0.55,0.67,0.87}6 & Write thesis \\
%    \end{tabular}
%    \caption{Research objectives}
%    \label{tab:objectives}
%  \end{center}
%\end{table}

%\begin{table}[h]
%  \begin{center}
%    \addtolength{\tabcolsep}{4pt} % new width
%    \renewcommand{\arraystretch}{0.8} % new height
%      \begin{tabular}{ l c c c c c c c }
%        && \textbf{1} & \textbf{2} & \textbf{3} & \textbf{4} & \textbf{5} & \textbf{6} \\ 
%        \multirow{4}{*}{\textbf{Feb}} & w1 & \cellcolor[rgb]{0.94,0.96,0.98} \\ & w2 & \cellcolor[rgb]{0.94,0.96,0.98} \\ & w3 & \cellcolor[rgb]{0.94,0.96,0.98} \\ & w4 & \cellcolor[rgb]{0.94,0.96,0.98} \\ 
%        \multirow{5}{*}{\textbf{Mar}} & w5 && \cellcolor[rgb]{0.86,0.90,0.96} \\ & w6 && \cellcolor[rgb]{0.86,0.90,0.96} \\ & w7 && \cellcolor[rgb]{0.86,0.90,0.96} \\ & w8 && \cellcolor[rgb]{0.86,0.90,0.96} \\ & w9 && \cellcolor[rgb]{0.86,0.90,0.96} \\ 
%        \multirow{4}{*}{\textbf{Apr}} & w10 &&& \cellcolor[rgb]{0.78,0.84,0.94} \\ & w11 &&& \cellcolor[rgb]{0.78,0.84,0.94} \\ & w12 &&& \cellcolor[rgb]{0.78,0.84,0.94} \\ & w13 &&& \cellcolor[rgb]{0.78,0.84,0.94} \\ 
%        \multirow{4}{*}{\textbf{May}} & w14 &&&& \cellcolor[rgb]{0.71,0.78,0.92} \\ & w15 &&&& \cellcolor[rgb]{0.71,0.78,0.92} \\ & w16 &&&& \cellcolor[rgb]{0.71,0.78,0.92} \\ & w17 &&&& \cellcolor[rgb]{0.71,0.78,0.92} \\ 
%        \multirow{5}{*}{\textbf{Jun}} & w18 &&&&& \cellcolor[rgb]{0.63,0.73,0.89} \\ & w19 &&&&& \cellcolor[rgb]{0.63,0.73,0.89} \\ & w20 &&&&& \cellcolor[rgb]{0.63,0.73,0.89} \\ & w21 &&&&& \cellcolor[rgb]{0.63,0.73,0.89} \\ & w22 &&&&& \cellcolor[rgb]{0.63,0.73,0.89} \\ 
%        \multirow{4}{*}{\textbf{Jul}} & w23 &&&&&& \cellcolor[rgb]{0.55,0.67,0.87} \\ & w24 &&&&&& \cellcolor[rgb]{0.55,0.67,0.87} \\ & w25 &&&&&& \cellcolor[rgb]{0.55,0.67,0.87} \\ & w26 &&&&&& \cellcolor[rgb]{0.55,0.67,0.87} \\ 
%      \end{tabular}
%    \caption{Time planning}
%    \label{tab:planning}
%  \end{center}
%\end{table}