\chapter{Introduction} 

To tackle climate change and its negative impacts, 
the historic Paris Agreement 
sets long-term goals to guide all nations 
to substantially reduce global greenhouse gas (\gls{ghg}) emissions 
to limit the global temperature increase
to 2 degrees Celsius in this century \cite{paris}. 
In 2019, 
80.9\% of our total energy supply still depended on burning fossil fuels, 
namely 
26.8\% coal, 30.9\% oil and 23.2\% natural gas \cite{iea}. 
However, 
investments into low-carbon power generation 
accounted for 15\% recently 
are expected to rise to more than 30\% by 2030, 
corresponding to a quadrupling in absolute volumes \cite{shift}. 
Solar, wind, and the investments 
for enabling the integration of these technologies to the grid 
dominate the investments into low-carbon power generation \cite{shift}. 

On European Union (\gls{eu}) level, 
“Energy 2020. A strategy for competitive, sustainable and secure energy”, 
published in November 2010,
and 
“Energy Roadmap 2050”, 
published at the end of 2011,
are the most important strategy papers currently,  
pointing the direction for energy developments in the \gls{eu},  
and aims to achieve decarbonisation goals 
of reducing \gls{ghg} emissions to 80-95\% below 1990 levels
by 2050 \cite{roadmap}.

\section{Problems}

Some text here ...

\section{Motivations}

Some text here ...

